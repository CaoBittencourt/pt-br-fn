\documentclass[12pt, a5paper, titlepage]{article}

\usepackage{pdfpages}
% \usepackage[backend=biber,style=apa]{biblatex}
\usepackage{tocloft}
\setlength{\cftsecindent}{0pt}
\setlength{\cftsubsecindent}{0pt}
\setlength{\cftsubsubsecindent}{0pt}

\usepackage{amsmath}
\usepackage{tipa}

\usepackage{tabularray}
\UseTblrLibrary{booktabs}

\usepackage[portuguese]{babel}
\setlength{\emergencystretch}{20pt}


% [task] extract to module
\newcommand{\textll}{\textsubbar{l}\textsuperscript{j}}

% \addbibresource{references.bib}
% \newcommand{\PtTableAbc}{
%     \begin{longtblr}[
%         caption = {Alfabeto português-brasileiro},
%         ]{
%         colspec = {X[c, m]X[c, m]X[c, m]X[c, m]X[c, m]X[c, m]},
%         width = \linewidth,
%         rowhead = 1,
%         rowfoot = 0
%         }
%         \toprule
%         Aa & Bb & Cc & Dd & Ee & Ff & Gg \\
%         Hh & Ii & Jj & Kk & Ll & Mm & Nn \\
%         Oo & Pp & Qq & Rr & Ss & Tt & Uu \\
%         Vv & Ww & Xx & Yy & Zz &    &    \\
%         \bottomrule
%         % \toprule
%         % Aa & Bb & Cc & Dd & Ee & Ff \\
%         % Gg & Hh & Ii & Jj & Kk & Ll \\
%         % Mm & Nn & Oo & Pp & Qq & Rr \\
%         % Ss & Tt & Uu & Vv & Ww & Xx \\
%         % Yy & Zz &    &    &    &    \\
%         % \bottomrule
%     \end{longtblr}
% }

\newcommand{\PtTableAbc}{
    \begin{longtblr}[
        caption = {Alfabeto português-brasileiro},
        ]{
        colspec   = {*{5}{X[c,m]}},
        width = \linewidth,
        row{odd}  = {font=\boldmath},
        row{odd}  = {cmd=\scalebox{1.25}},
        row{even} = {cmd=\scalebox{0.75}},
        row{odd}  = {abovesep+=5pt},
        row{even}  = {abovesep+=-5pt},
        row{even}  = {belowsep+=7pt},
        rowhead = 1,
        rowfoot = 0
        }
        \toprule
        Aa & Bb & Cc & Dd & Ee & Ff & Gg \\
        {[a]} & {[be]} & {[se]} & {[de]} & {[e]} & {[\textepsilon fi]} & {[\textyogh e]} \\
        
        Hh & Ii & Jj & Kk & Ll & Mm & Nn \\
        {[ag\textprimstress{a}]} & {[i]} & {[\textyogh \textopeno ta]} & {[ka]} & {[]} & {[Mm]} & {[Nn]} \\
        
        Oo & Pp & Qq & Rr & Ss & Tt & Uu \\
        Oo & Pp & Qq & Rr & Ss & Tt & Uu \\
        
        Vv & Ww & Xx & Yy & Zz &    &    \\
        Vv & Ww & Xx & Yy & Zz &    &    \\
        \bottomrule
    \end{longtblr}
}

\newcommand{\BrTableAbc}{
    \begin{longtblr}[
        caption = {Awfabétu brazileyru},
        ]{
        colspec   = {*{5}{X[c,m]}},
        width = \linewidth,
        row{odd}  = {font=\boldmath},
        row{odd}  = {cmd=\scalebox{1.25}},
        row{even} = {cmd=\scalebox{0.75}},
        row{odd}  = {abovesep+=5pt},
        row{even}  = {abovesep+=-5pt},
        row{even}  = {belowsep+=7pt},
        rowhead = 1,
        rowfoot = 0
        }
        \toprule
        Aa & Bb & Cc & Dd & Ee & Ff \\
        {[a]} & {[be]} & {[ka]} & {[de]} & {[e]} & {[\textepsilon fi]} \\
        
        Gg & Hh & Ii & Yy & Jj & Ll \\
        {[ga]} & {[\textepsilon hi]} & {[i]} & {[kwazi i]} & {[\textyogh \textopeno ta]} & {[\textepsilon \textturny i]} \\
        
        Mm & Nn & Oo & Tt & Pp & Rr \\
        {[emi]} & {[eni]} & {[o]} & {[te]} & {[pe]} & {[\textepsilon \textfishhookr i]} \\
        
        Ss & Uu & Ww & Vv & Xx & Zz \\
        {[\textepsilon si]} & {[u]} & {[kwazi u]} & {[ve]} & {[\textesh is]} & {[ze]} \\
        \bottomrule
    \end{longtblr}
}

% \newcommand{\BrTableAbc}{
%     \begin{longtblr}[
%         caption = {Awfabétu brazileyru},
%         ]{
%         colspec = {X[c, m]X[c, m]X[c, m]X[c, m]X[c, m]X[c, m]},
%         width = \linewidth,
%         rowhead = 1,
%         rowfoot = 0
%         }
%         \toprule
%         Aa & Bb & Cc & Dd & Ee & Ff \\
%         Gg & Hh & Ii & Yy & Jj & Ll \\
%         Mm & Nn & Oo & Tt & Pp & Rr \\
%         Ss & Uu & Ww & Vv & Xx & Zz \\
%         % Aa & Bb & Cc & Dd & Ee & Ff \\
%         % Gg & Hh & Ii & Yy & Jj & Ll \\
%         % Mm & Nn & Oo & Pp & Rr & Ss \\
%         % Tt & Uu & Ww & Vv & Xx & Zz \\
%         \bottomrule
%     \end{longtblr}
% }

\newcommand{\PtTableVowels}{
    \begin{longtblr}[
        caption = {Pronúncia padrão},
        ]{
        colspec = {X[0, c, m]X[1, c, m]X[0.5, c, m]},
        width = \linewidth,
        rowhead = 1,
        rowfoot = 0
        }
        Letra & Pronúncia padrão & Exemplo \\
        \toprule
        Aa    & Agudo            & dsds    \\
        Ee    & Grave            & dsds    \\
        Ii    & Agudo            & dsds    \\
        Oo    & Grave            & dsds    \\
        Uu    & Agudo            & dsds    \\
        Yy    & Agudo            & dsds    \\
        Ww    & Agudo            & dsds    \\
        \bottomrule
    \end{longtblr}
}

\newcommand{\PtTableDiacritics}{
    \begin{longtblr}[
        caption = {Acentos da língua brasileira},
        label = {pt.table.diacritics}
        ]{
        colspec = {X[0, c, m]X[1, c, m]X[0.5, c, m]},
        width = \linewidth,
        rowhead = 1,
        rowfoot = 0
        }
        Acento             & Nome                 & Exemplo \\
        \toprule
        \textasciiacute    & Acento agudo         & dsds    \\
        \textasciigrave    & Acento grave         & dsds    \\
        \textasciitilde    & Acento nasal         & Princípio    \\
        \textasciicircum   & Acento nasal forte   & Príncipe    \\
        \textasciidieresis & Acento duplo (crase) & Àquela    \\
        \bottomrule
    \end{longtblr}
}

\newcommand{\PtTableDigraphs}{
    \begin{longtblr}[
        caption = {Dígrafos},
        note{1} = {Dependendo se o ``i" for semivogal ou não.},
        note{2} = {Pós-vocálico, ou ``r'' seguido de vogal e consoante.},
        note{3} = {A palavra ``restaurador'' é interessante, porque contém todos os ``r'' brasileiros: o primeiro é gutural; o segundo, intervocálico; e o terceiro, pós-vocálico.}
        ]{
        colspec = {X[0, c, m]X[0, c, m]X[1, c, m]},
        width = \linewidth,
        rowhead = 1,
        rowfoot = 0
        }
        Antiga Grafia              & Nova Grafia           & IPA                           & Exemplo     \\
        \toprule
        ss                         & s                     & [s]                           &             \\
        sc                         & s                     & [s]                           &             \\
        sç                         & s                     & [s]                           &             \\
        xs                         & s                     & [s]                           &             \\
        xc                         & s                     & [s]                           &             \\
        qu                         & cw                    & [kw]                          & qualidade   \\
        qu                         & c                     & [k]                           & queijo      \\
        gu                         & gw                    & [gw]                          & aguenta     \\
        gu                         & g                     & [g]                           & guerra      \\
        lh                         & lly                   & [?]                           &             \\
        nh                         & \~y, ĩ, î\TblrNote{1} & [\~j]                         &             \\
        ch                         & x                     & [\textesh]                    &             \\
        rr                         & h                     & [h]                           &             \\
        r\TblrNote{2}              & hr                    & [], [], []\TblrNote{3}        & restaurador\TblrNote{3} \\
        di                         & dji                   & [\texttoptiebar{d\textyogh}i] &             \\
        ti                         & txi                   & [\texttoptiebar{t\textesh}i]  &             \\
        li                         & lli, lly\TblrNote{1}  & [?i]                          &             \\
        \bottomrule
    \end{longtblr}
}


\title{Manual da língua brasileira para luzófonos}
\author{Cao Bittencourt}
\date{}

\begin{document}

\maketitle
\thispagestyle{empty}

\newpage \
\thispagestyle{empty}

\newpage
\tableofcontents
\newpage

\section{Introdução}
[introdução 1]
\newpage
[introdução 2]

\newpage
\section{Alfabeto}\label{pt.section.abc}
Comecemos pelo mais básico, o alfabeto:

\PtTableAbc
Conforme as tabelas, o novo alfabeto brasileiro (à esquerda) tem vinte e quatro letras, enquanto o antigo alfabeto português-brasileiro (à direita) tem vinte e seis. As letras removidas foram o ``k'' e o ``q'', porque são redundantes. De fato, a primeira delas já era até na antiguidade clássica criticada pelos gramáticos romanos, que achavam-na desnecessária. A letra ``q'', por sua vez, foi uma invenção desses mesmos gramáticos para diferenciar o som do ``u'' vogal e do ``u'' semivogal (cf. as palavras \textit{qui} [kwi] e \textit{cui} [ku\textprimstress i]). Esse mecanismo linguístico, no entanto, é evidentemente rudimentar -- como outras de suas invenções (e.g. os algarismos romanos) --, pois não permite distinguir entre vogais e semivogais nas palavras sem ``q'', que são muito frequentes em nossa língua. Além disso, na maioria das palavras brasileiras \textit{com} ``q'' o ``u'' não é nem sequer semivogal, mas mudo e inexistente (a exemplo do próprio ``quê'' [ke] utilizado nesta mesma frase, que [ki] vem do ``qui'' latino). Assim sendo, removemos do alfabeto aquela [a\textprimstress k\textepsilon la] letra, desprezada pelos romanos, e, ironicamente, também essa, que inventaram.

Não há novas letras no alfabeto, porém muitas das que permaneceram passam a ter
novas funções; e, mais importante, uma única função para cada. A letra ``c'',
por exemplo, para continuar a discussão acima, tem agora sempre o som de ``k'',
nunca de ``s''; na verdade, foi até renomeada para ``Cá'' [ka], a fim de deixar
isso mais claro. Pelo mesmo motivo, o ``Cê-cedilha'', ``ç'', é substituído por
``s''. E, com isso, acaba-se a milenar confusão entre as consoantes oclusiva velar
surda [k] e a fricativa alveolar surda [s], que herdamos do latim.

Analogamente, a letra ``g'' é renomeada para ``Gá'' [ga] e conserva apenas uma de suas funções anteriores. Com efeito, o nome português, ``Gê'' [\textyogh e], já [\textyogh a] ilustrava, por si só, sua ambiguidade fonética; e em palavras como ``garagem'' [ga\textprimstress\textfishhookr a\textyogh \~e\~\j], ainda, verificava-se o mesmo ``g'' com duas pronúncias completamente diferentes. Agora esse problema foi resolvido; e o novo nome representa isso muito bem, porquanto, embora o antigo ``Gê'' fosse ambíguo, todos concordam que ``g'' seguido de ``a'' pronuncia-se com a consoante oclusiva velar sonora [g], e não com um ``j'' de francês.

Seguindo a ordem alfabética, o ``Agá'' deixa de ser uma letra mal utilizada, essencialmente inútil, e assume o som fricativo glotal surdo [h], ou ``Erre'' gutural, como é nos demais idiomas da Europa (e.g. nas palavras \textit{home}, \textit{heim} e \textit{hjem}, ou seja, ``lar'' em inglês, alemão e norueguês, respectivamente). Isso significa que todas as palavras que começavam com ``r'' ou tinham dois ``r'' (e.g. ``rato'' e ``torra'') passam a ser escritas com ``h''; e, consequentemente, o ``r'' é reservado para o tepe alveolar (o fonema em ``arara''). Por fim, removem-se todos os ``h'' mudos; e, como nas outras letras, renomeia-se o ``Agá'' para ``Erre'' [\textprimstress\textepsilon hi] e o ``Erre'' para ``Eri'' [\textprimstress\textepsilon\textfishhookr i], sinalizando suas novas funções.

Ao contrário das supracitadas línguas germânicas, entretanto, o ``j'' conserva a pronúncia que recebemos dos franceses, não sendo utilizado para o som de ``i'' semivogal (como vimos em \textit{hjem}, acima).

Esse som, cujo fonema denota-se por [j], pertence ao ``y'', que, de maneira análoga ao ``h'', antes subutilizado, torna-se uma letra muito importante, uma vez que o brasileiro é um idioma repleto de semivogais.

Portanto, a indicação das semivogais não é nem negligenciada, como vinha sendo desde o Acordo Ortográfico de 1990, tampouco se dá pelo contexto ou o antiquado ``Trema''. Em contraposição, a nova língua brasileira designa letras específicas para esse fim, a saber, o ``y'', chamado ``Quasi-i'', e o``w'', ou ``Quasi-u''.

Não é necessário mais letras do que essas, porque apenas o ``i'' e o ``u'' são semivogais, enquanto o ``a'', o ``e'' e o ``o'' são sempre vogais (i.e. elas ``quebram'' a sílaba e não aglutinam-se em ditongos e tritongos).

As semivogais, no entanto, também podem substituir (aparentes) consoantes. É o que verificamos no ``l'' pós-vocálico, que no português brasileiro tem o som do ``u'' semivogal [w], e não o do ``l'' propriamente dito, isto é, a aproximante lateral alveolar [l], como no alemão \textit{hilfe} [\textprimstress h\textsci lf\textschwa], ou no português europeu, ``fiel'' [fi\textprimstress\textepsilon l], com ``l'' pronunciado (cf. no brasileiro, [fi\textprimstress \textepsilon w], onde não tem som consonantal).

E idem com ``n'' e ``m'' pós-vocálicos, que dão lugar a ``y'' e ``w'' acentuados com o acento nasal, devendo-se à característica distintiva da nossa língua de que vogais seguidas de ``n'' e ``m'' (com uma consoante depois) produzem um som semivocálico residual, não capturado por essas consoantes, mas sim pelas semivogais nasalizadas (``\~y'', ``\~w''), conforme ilustrado em [\~e\~\j]

[exemplos]

Praticamente todos os outros idiomas escritos com o alfabeto latino, porém, não possuem essa ``semivogal residual'', então é correto utilizarem o ``n'' e o ``m'' pós-vocálicos (e.g. [exemplos]). Mas, como o nosso objetivo é que o \textit{brasileiro} seja consistente, devemos substituí-los por semivogais nasalizadas.

Finalmente, as duas últimas letras, ``x'' e ``z'' representam, cada, um único fonema e não mais se confundem entre si nem com ``s'' e ``c''. Especificamente, o ``x'' deixa de ter som de [ks], [s], [z] e mantém-se apenas como fricativa pós-alveolar surda [\textesh], o mesmo que o antigo ``ch''. Já o [z] é grafado por ``z'', inclusive nas palavras com ``s'' intervocálico (e.g. ``casa''); e não há mais ``z'' no final de nenhuma palavra. Desse modo, todas as letras no alfabeto têm sua própria função.

\section{Vogais}
Como aludido acima, as vogais na língua brasileira são ``a'', ``e'', ``i'', ``o'', ``u''; e as semivogais, ``y'' e ``w'' (``Quasi-i'' e ``Quasi-u''). As vogais formam hiátos se adjacentes, mas as semivogais aglutinam-se. Assim, nos encontros vocálicos

\begin{center}
    \begin{tabular}{ll}
        ``(em pt-br) palavra com vogal + w'' &[IPA],
        \\
        ``(em pt-br) palavra com y + vogal'' &[IPA],
        \\
        ``(em pt-br) palavra com y + vogal + w'' &[IPA],
        \\
        ``(em pt-br) palavra com w + vogal + w'' &[IPA],
    \end{tabular}
\end{center}
as semivogais e as vogais formam uma única sílaba; já nos hiatos

\begin{center}
    \begin{tabular}{ll}
        ``(em pt-br) exemplo de hiato'' &[IPA],
        \\
        ``(em pt-br) exemplo de hiato'' &[IPA],
        \\
        ``(em pt-br) exemplo de hiato'' &[IPA],
    \end{tabular}
\end{center}
cada vogal é pronunciada separadamente.

Ademais, porque visamos a consistência fonética (i.e. que se escreva como se diz), convém distinguir não só entre vogais e semivogais, mas ainda entre as agudas, graves e nasais.

Denotá-las explicitamente exigiria ou uma letra para cada som (como é no Alfabeto Fonético Internacional) ou algum sistema de acentuação. A primeira opção não seria nem um pouco prática, e a segunda também pode tornar-se trabalhosa se não implementada direito.

\subsection{Acentos}
Em particular, para evitar excessivos acentos, opta-se por não explicitar graficamente quando o fonema é oposto ao padrão\footnote{Via de regra, são agudas as letras ``a'', ``i'', ``u'', ``y'' e ``w''; e graves o ``e'' e ``o''.} (viz. o mais frequente), reservando os diacríticos sobretudo para indicar tonicidade (som fraco ou forte), não a altura (som agudo ou grave).

Certamente, trata-se de uma decisão bastante lamentável, considerando o objetivo desta reforma. No entanto, é o único caminho factível, por estas razões.

Em primeiro lugar, nas palavras paroxítonas, que são as mais comuns, há uma tendência a abrir as vogais fechadas (viz. o ``e'' e o ``o''), mas o padrão não é claro o suficiente para estabelecerem-se regras de acentuação. Comparem-se, por exemplo, as palavras

\begin{center}
    \begin{tabular}{rcl}
``mesa'' [\textprimstress meza] &e &[\textprimstress m\textepsilon ta] ``meta'';\\
``mesma'' [\textprimstress mesma] &e &[\textprimstress m\textepsilon scla] ``mescla'';\\
``colha'' [\textprimstress ko\textll a] &e &[\textprimstress k\textopeno la] ``cola'';\\
``corra'' [\textprimstress koha] &e &[\textprimstress k\textopeno hi] ``corre'';\\
``beleza'' [be\textprimstress leza] &e &[\textprimstress b\textepsilon la] ``bela'';\\
``gosto'' [\textprimstress gostu] &e &[\textprimstress g\textopeno stu] ``gosto'';\\
``posto'' [\textprimstress postu] &e &[\textprimstress p\textopeno stu] ``posto''.\\
    \end{tabular}
\end{center}

Existem inúmeras paroxítonas como essas e, entretanto, nenhuma regra infalível para diferenciá-las. Se houvesse, talvez seria tão simples quanto distinguir
\begin{center}
    \begin{tabular}{rcl}
``ôca'' [\textprimstress oka] &de &[\textprimstress \textopeno ka] ``oca'',\\
    \end{tabular}
\end{center}
bastando uma ser acentuada, e a outra não.

Mas, infelizmente, o português é mais complexo; e tão complexo é, inclusive, que conta com múltiplas variantes regionais (cf. as vogais abertas nos sotaques do nordeste e as fechadas no sudeste), um outro motivo para evitarmos indicar altura com acentos, porquanto fazê-lo fragmentaria ainda mais o idioma.

% É (quase) unânime no Brasil, porém, pronunciar-se grave o ``a'' seguido de ``n'' ou ``m'', principalmente quando tônico (cf. ``ano'' [\textprimstress\~\textturna nu] e anotação [anota\textprimstress s\~\textturna\~w]), com relativamente poucas exceções. Se fôssemos explicitar todo ``a'' grave, portanto, teríamos de acentuar tantas palavras que tornaria-se enfadonho.
É (quase) unânime no Brasil, porém, pronunciar-se grave o ``a'' seguido de ``n'' ou ``m'', principalmente quando tônico (cf. ``ano'' [\textprimstress\~\textturna nu] e anotação [anota\textprimstress s\~\textturna\~w]), com relativamente poucas exceções.

Mas, se fôssemos explicitar todo ``a'' grave, teríamos de acentuar tantas palavras que tornaria-se enfadonho. De fato, devido à função dual dos diacríticos em português, o acento grave indicando a altura do ``a'' poderia ser interpretado como indicando tonicidade; e, para desambiguar, seriam necessários mais acentos ainda.

Um ótimo exemplo é a palavra ``mamãe'': porque é oxítona\footnote{Ver regras de acentuação adiante.} leva acento grave (porque o ``a'' tônico é grave); mas o primeiro ``a'' também é grave, e não é tônico. Assim, precisaríamos: ou de um acento duplo-grave, como nas línguas croata e eslovena; ou deixaríamos como está (i.e. sem qualquer indicação); ou convencionaríamos que o ``a'' antes de ``m'' é sempre grave (ou uma regra equivalente).

Ora, eis que a palavra ``mamão'' tem exatamente a mesma estrutura, e o primeiro ``a'' é agudo: então, a última solução não serve. E, obviamente, a primeira não é prática. Resta deixarmos como está. E o mesmo verifica-se em: [exemplos]. Há outros exemplos, mas isso ilustra o ponto o suficiente.

Um quarto motivo refere-se ao plural metafônico, fenômeno tão comum em português de o singular e o plural terem pronúncias distintas (e.g. ``olho'' [] e ``olhos'' [], ``novo'' [] e ``novos'' [], e assim por diante). Por si só, a metafonia não é o bastante para remover os diacríticos de altura, pois poderiam-se convencionar regras de pronúncia, conquanto até essas regras teriam exceções; por exemplo, ``restolhos'' (cf. ``olhos'', acima), ou ``seu'' e ``seus'', onde não há metafonia. Porém, somado aos demais fatores, essa parece ser uma boa razão para reservar os acentos a outras funções.

Além disso, a fidedignidade fonética de um sistema como este que aqui pretendemos inevitalmente acaba por produzir mais palavras homógrafas; e, por consequência, não acentuar sempre que a pronúncia difere do padrão torna-se até conveniente para distinguir o que, de outro modo, seria idêntico.

Como exemplo, comparem-se ``olho'', de novo, e ``óleo'' []: logo percebe-se que, se quisermos que escrevam-se como nós, brasileiros, as pronunciamos, e não como a grafia lusitana sugere (literalmente, []), não teríamos como diferenciá-las senão pelo acento agudo. E idem para [exempos], etc.

Por fim, deve-se considerar a praticidade desta reforma ortográfica nos tempos atuais. Pois, embora continue-se a escrever bastante em papel e caneta, se comparado ao passado, hoje é mais importante a facilidade da escrita no teclado de um computador; e o que é rápido à mão livre pode não sê-lo para digitar.

As línguas eslavas mencionadas acima são um caso extremo disso. Nelas, estima-se que digitar demore em torno de 25\% a mais ([citar]), devido à grande quantidade de diacríticos tanto em vogais quanto em consoantes (cf. ``Vou malhar hoje''\footnote{
    A versão tradicional dessa frase tem apenas uma letra a menos, e a quantidade de caracteres a digitar até seria igual, não fosse o ``hr'', cuja \textit{raison d'être} é linda (vide o Capítulo \ref{pt.section.digraphs}).
} e o polonês, \textit{Zamierzam dziś poćwiczyć}). No português brasileiro fonético, queremos evitar esse problema. Por isso, convém reduzir os acentos sempre que não prejudicar o entendimento.

% [mencionar em algum lugar que semivogais explícitas removem as palavras esdrúxulas (água -> agwa, etc)]

\subsubsection{Os novos diacríticos da língua brasileira}
Feitas essas muitas ressalvas, destaca-se que os acentos ainda assim continuam a sinalizar pronúncias alternativas (sejam elas aguda, grave ou nasal), modificando o som padrão de vogais e semivogais, além de diferenciarem palavras idênticas\footnote{
    As homógrafas de que falamos, bem como as homófonas, também comuns. Sendo muito simples a função de diferenciar palavras de mesmo som, opta-se por não explicá-la em detalhes, para não interromper o fluxo deste texto. Porém, a título de exemplo, comparem-se o verbo ``há'' e o artigo ou a preposição ``a''. Fica claro, aqui, que tendo removido o ``h'' mudo (vide o Capítulo \ref{pt.section.abc}), o acento é o único meio de distingui-las.
} e indicarem a sílaba tônica se não for autoevidente, como já discutido. Porém, apesar de essas funções serem iguais às de antes, a maioria dos diacríticos, em si, mudaram significativamente.

O acento agudo permanece o mesmo, mas adquire agora sua contraparte ``mais simétrica'' no acento grave, ao invés do circunflexo: o acento ``ascendente'' denota vogais abertas e o ``descendente'', vogais fechadas (ou seja, o exato oposto do italiano, do francês e do espanhol). Então, quando o acento é agudo, a sílaba torna-se tônica e a pronúncia, aguda, mesmo que a pronúncia esperada da vogal fosse grave; e analogamente, se o acento for grave.

\PtTableDiacritics

O acento nasal também serve para indicar uma pronúncia alternativa. Entretanto, nisso difere bastante de como costumava ser. Na língua brasileira, o diacrítico ``\textasciitilde'', não mais chamado ``Til'', e sua versão tônica\footnote{
    Não há diferença na pronúncia dos dois diacríticos, apenas na tonicidade: o acento nasal forte indica a sílaba tônica, além de nasalização, enquanto o nasal fraco é átono.
}, ``\textasciicircum'', faz com que a vogal seja articulada como seria se fosse seguida de ``n'' ou ``m'', porém de uma maneira inteiramente vocálica, ``torcendo'' o som com o nariz, sem a obstrução física que caracteriza as consoantes. Isto é, não trata-se de uma vogal ``tendendo'' ao ``n'' ou ``m'', como é no inglês ou no italiano, onde a consoante é pronunciada, mas daquele som nasal \textit{não consonantal}, que é marca do português brasileiro:

\begin{center}
    \begin{tabular}{ll}
        ``exemplo'' &[IPA];
        \\
        ``exemplo'' &[IPA];
        \\
        ``exemplo'' &[IPA];
        \\
        ``exemplo'' &[IPA].
    \end{tabular}
\end{center}

% [parágrafo explicando em termos técnicos nasal-palatal]
Acrescenta-se, ainda, que uma vogal nasalizada pode ser tanto aguda quanto grave (cf. \textit{adelante} [] em espanhol e ``adiante'' [] em português). Em teoria, isso requeriria acentos mais específicos, mas, convenientemente, a pronúncia das vogais nasalizadas é consistente na língua brasileira: o ``a'' nasal é sempre grave; e ``i'', ``y'', ``u'' e ``w'' nasais são sempre agudos. 

Já o ``e'' e o ``o'' nunca são nasalizados diretamente, porque o som que produziriam, de acordo com a nova definição do acento nasal, não ocorre no português brasileiro. Dito isso, se eram seguidos de ``n'' ou ``m'', passam a acompanhar ``\~y'' e ``\~w'', de novo, por causa da semivogal residual implícita nesses dígrafos.

E, finalmente, também concernindo às semivogais, note-se na Tabela \ref{pt.table.diacritics} (especificamente, em ``princípio'') que não se fazem mais necessários os diacríticos nas palavras esdrúxulas -- a razão sendo que esses acentos eram senão uma (insatisfatória) compensação para a ausência de semivogais explícitas no português. Agora que temos o ``Quasi-i'' e o ``Quasi-u'', ninguém ficaria em dúvida de que a palavra ``água'' [\textprimstress agwa], por exemplo, não pronuncia-se ``agúa'' [a\textprimstress gua]. Portanto, todas as ``proparoxítonas'' esdrúxulas perdem o acento e convertem-se em paroxítonas verdadeiras.

% \subsubsection{Tonicidade e acentuação}
\subsubsection{Regras de acentuação e tonicidade}
Tendo entendido isso, passamos para a terceira função dos acentos, qual seja, a indicação da sílaba tônica.

Aqui, não difere-se muito do português: na língua brasileira, há apenas palavras oxítonas, paroxítonas e proparoxítonas, cujas sílabas tônicas são, respectivamente, a última, a penúltima e a antepenúltima.

Dentre essas as mais frequentes são as paroxítonas. E, assim sendo, convenciona-se, \textit{ceteris paribus}, que toda palavra de mais de uma sílaba seja paroxítona e não necessita de acento.

Já as proparoxítonas são, de longe, as mais raras e, por esse motivo, permanecem sempre acentuadas.

As oxítonas, por sua vez, são relativamente comuns, mas nem sempre requerem acento; isso porque existem alguns padrões confiáveis para identificá-las.

Desse modo, se não indicado explicitamente, são oxítonas (e não acentuadas) todas as palavras terminadas em: ``hr'', ``e'', ``o'', ``y'', ``w'', ``o\~y'', ``o\~w'' (e, igualmente, os plurais, acrescidos de ``s'' no final).

Enfim, com respeito à ``hierarquia'' dos diacríticos, a regra é bem simples: os acentos nasal forte, agudo e grave são tônicos (i.e. onde aparecerem torna-se automaticamente a sílaba tônica); o acento nasal, embora possa coincidir com a penúltima sílaba em paroxítonas, não é, por si só, tônico; e, analogamente, a crase é átona, pois somente consiste na junção de um ``a'' preposição com um ``a'' artigo\footnote{
    Motivo pelo qual é grafada com o ``acento duplo'', podendo também ser substituída por dois ``a'' quando conveniente.
}, e não modifica nem a pronúncia nem a tonicidade do ``a'' que acentua.

\subsection{Lei da gravidade vocálica}
Antes de concluir, [convém] reconhecermos algo que certamente o leitor já percebeu a essa altura, a saber, de que praticamente todas as palavras terminadas em ``e'' passaram a ser escritas com ``i'' e as terminadas em ``o'', com ``u''. Esse tópico foi removido da exposição do alfabeto por brevidade e por ser mais pertinente a essa seção (das vogais). E, assim, sendo tão importante, devemos retomá-lo agora.

É amplamente documentado que o português brasileiro sofre do processo de redução vocálica.
Aqui, o fenômeno não é tão intenso quanto em Portugal, onde chega-se a eliminar vogais inteiras se não tônicas, entretanto ainda faz com que as pronunciemos de uma maneira bem diferente do que a escrita sugere.

O caso mais frequente no Brasil é de quando o ``o'' átono [\textprimstress atonu] tem som de ``u'' no [nu] final das palavras. Nisso [\textprimstress nisu], há pouquíssima divergência regional: apenas raramente encontra-se quem conserve um sotaque ``antiquado'' o suficiente a ponto de continuar a falar ``do jeito correto'' [co\textprimstress\textfishhookr\textepsilon to]; praticamente todos os brasileiros já divergiram dessa pronúncia ``arcaica''.

Permanece (algo) controverso, no entanto, trocar-se ``e'' fraco por ``i'', especialmente no Sul, em que muitos mantêm o fonema [e], mesmo na última sílaba. Isso varia de cidade [si\textprimstress dade] em cidade [si\textprimstress da\texttoptiebar{d\textyogh}i], afinal algumas, bem populosas e em contato constante com falantes de outros sotaques, estão sujeitas a maior ``erosão fonética'' do que outras, mais isoladas; mas, ainda assim, é fácil na região ouvir-se falar, digamos, ``leite'' [\textprimstress lejte] com ``e'', ao invés do mais comum [\textprimstress lej\texttoptiebar{t\textesh}i].

Porém, questiona-se por quanto tempo ``sobreviverá'' essa pronúncia; e isso não só por conta da mencionada ``erosão vocálica'': acredita-se que o fenômeno seja natural e que todos os sotaques eventualmente convergirão nesse aspecto. Em outras palavras, considerando a fonologia brasileira, a redução dessas vogais é provavelmente inevitável, independente do sotaque.

A razão é o que aqui resolvemos denominar de ``lei da gravidade vocálica''. Resumidamente, a ``lei da gravidade vocálica''\footnote{
    Ou não necessariamente lei, talvez só uma hipótese.
} diz que todo ``e'' fraco tende a ``i'' e todo ``i'' fraco tende a ``Quasi-i''\footnote{
    Idem para o ``o'', ``u'' e ``Quasi-u''.
}. Eis o porquê.

Como em Física, na Fonética, toda palavra tem por ``centro de massa'' o ponto de articulação principal, sua sílaba tônica. E para esse ``centro de massa'' conduzem os demais fonemas: é na sílaba tônica onde a palavra finalmente ``acontece''; e as sílabas anteriores têm importância somente em levar-nos a ela; as posteriores  enfatizando-se até menos do que essas (motivo pelo qual, ao escreverem-se versos, se descartarem da contagem de sílabas poéticas aquelas após a tônica).

Assim, perdem a força as vogais antes e depois do ``centro de massa''. E isso, na fala cotidiana, rápida, dá a impressão de estarmos ``atropelando'' sua articulação, quase com ``pressa'' de chegar logo na sílaba tônica. É como se tudo fosse ``puxado'' pela força da gravidade na direção desse ``centro de massa''.

O efeito, ao longo dos séculos, é a redução vocálica: as vogais não tônicas degeneram a suas correlatas, tendo o ``e'' por correspondente o ``i''; e [i] o [u] ``o'' [o], o [u] ``u'' [u]; e, analogamente, para o ``Quasi-i'' e ``Quasi-u''; enquanto o ``a'', por sua vez, não tem qualquer paralelo entre as outras vogais e semivogais.

Na palavra ``linguagem'' temos um exemplo histórico interessante: originalmente escrita
\begin{center}
    \begin{tabular}{c}
        ``Lingoagem'',
    \end{tabular}
\end{center}
pronunciava-se com o fonema [o], vocálico, por volta dos anos 1536; com o passar do tempo, tornou-se
\begin{center}
    \begin{tabular}{c}
        ``linguagem'',
    \end{tabular}
\end{center}
com [u], também vogal; e, hoje, é ainda ``linguagem'', mas é dita com a semivogal [w]. Evidencia-se, então, que o ``o'' fraco tende a ``u''; e o ``u'' fraco, a ``Quasi-u''.

As palavras esdrúxulas, de que falamos, são mais um exemplo. Pois, de novo, não havendo, no português, diferenciação explícita entre [i], [j], [u] e [w], achou-se [inteligente] acentuá-las como paroxítonas, para que, mesmo quando pronunciadas com vogais longas, ao invés de [j] ou [w], a sílaba tônica não mude.

Dito isso, há, hoje em dia, uma pessoa sequer que as diga com vogais longas? Isto é, apesar de sua grafia deixar, em algum grau, ambígua [ã\textprimstress bigwa] a pronúncia, por acaso alguém diz [ã\textprimstress bigua], estendendo o ``u''? É claro que não. Até soa um tanto pretensioso, praticamente um sotaque aristocrático, hostil aos ouvidos modernos. Ninguém tem tempo para vogais longas numa linguagem dita às pressas. Por isso, toda e qualquer palavra esdrúxula degenera à sua versão semivocálica.

o ``i'' e o ``u'' são as semivogais por excelência. Realmente, muitos devem ter lido essa frase como ``u i i u u'' e rido, porque ``soa chinês''.


e $\rightarrow$ i $\rightarrow$ y; o $\rightarrow$ u $\rightarrow$ w
- o "i" e o "u" são as semivogais por excelência
    - de fato, muitos devem ter lido essa frase como "u i i u u" (e caído na gargalhada)
    - isso, por si só, já exemplifica a "Lei da Gravidade Vocálica"
- isto é, todo "e" fraco tende a "i" e todo "o" fraco tende a "u"
- com o passar do tempo, a "Lei da Gravidade Vocálica", por assim dizer, "puxa" o "e" fraco para o "i" e o "o" fraco para o "u"
- por isso, no latim, por exemplo, é tão comum as palavras terminarem em "us", e "is", etc.
- portanto, sabendo que, no longo prazo, a língua tende ao mesmo estado de maturidade que o latim, já nos adiantamos em substituir todo "o" fraco por "u" e todo "e" fraco por "i"
- aliás, hoje em dia, a língua portuguesa já está em seu estado de maturidade, porquanto já pronunciamos o "e" fraco e "o" fraco de acordo com a Lei da Gravidade Vocálica. convém, portanto, escrevermos de acordo também; assim, a língua se ajusta aos falantes, e não o contrário.

\subsection{Resumo dos fonemas vocálicos}
Explicadas as vogais e semivogais, podemos resumi-las citando alguns exemplos:

\begin{center}
    \begin{tabular}{ll}
        ``palavra'' &[IPA];
        \\
        ``não precisa de 1 exemplo por fonema'' &[IPA].
    \end{tabular}
\end{center}

E esses são todos os sons vocálicos no português brasileiro fonético.

\section{Dígrafos}\label{pt.section.digraphs}
Após a apresentação das vogais, só resta tratar dos dígrafos consonantais\footnote{
    Novamente, os antigos dígrafos vocálicos (viz. vogal seguida de ``n'' ou ``m''), foram substituídos por vogais nasalizadas, ou vogais seguidas de ``\~y'' ou ``\~w'', quando resultam em semivogal residual, como explicado no capítulo anterior.
}:

\PtTableDigraphs

A tabela acima compara os dígrafos no português com sua nova grafia na língua brasileira. Cita-se um exemplo de cada, junto à transcrição para o Alfabeto Fonético Internacional. Comentemo-los abaixo.

Primeiramente, os dígrafos ``ss'', ``sc'', ``sç'', ``xs'', ``xc'', referentes à consoante fricativa alveolar surda, deixam de sê-lo e são substituídos diretamente por ``s''.
% Primeiramente, os dígrafos referentes à consoante fricativa alveolar surda (em português, ``ss'', ``sc'', ``sç'', ``xs'', ``xc'') deixam de sê-lo e são substituídos diretamente pela letra ``s''.

Os dígrafos da ``consoante'' aproximante labiovelar [w], antes ambíguos (cf. na tabela,   ``aguenta'' e ``guerra''), sendo ora pronunciados, ora mudos, agora não são mais utilizados. No português brasileiro fonético, não há vogais mudas; e todas as semivogais são explícitas. Portanto, esses dígrafos não têm mais sentido: se o ``u'' era mudo, é removido; e, se era semivogal, passa a ser ``Quasi-u''.

% Quanto aos dígrafos da aproximante palatal [j], destaca-se, como já reconhecido pelos fonólogos (e.g. [citar]), que a língua brasileira não dispõe de um som aproximante lateral palatal [\textturny] propriamente dito. E idem para a consoante nasal palatal [\textltailn], isto é, o ``nh'' no português europeu, o ``gn'' em francês e italiano e o ``ñ'' espanhol (cf.
% Quanto aos dígrafos da aproximante palatal [j], destaca-se, como já reconhecido pelos fonólogos (e.g. [citar]), que o brasileiro não dispõe de um som aproximante lateral palatal [\textturny] propriamente dito. E idem para a consoante nasal palatal [\textltailn], isto é, o ``nh'' no português europeu, o ``gn'' em francês e italiano e o ``ñ'' espanhol (cf.
Quanto aos dígrafos da aproximante palatal [j], destaca-se, como já reconhecido pelos fonólogos (e.g. [citar]), que não dispomos de um som aproximante lateral palatal [\textturny] propriamente dito. E idem para a consoante nasal palatal [\textltailn], isto é, o ``nh'' no português europeu, o ``gn'' em francês e italiano e o ``ñ'' espanhol (cf.
``montanha'', \textit{montagne}, \textit{montagna}, \textit{montaña}). A diferença é sutil, mas notável.

No Brasil, a nossa pronúncia é muito mais vocálica: aqui, o que temos, no lugar dessas consoantes, são a aproximante lateral alveolo-palatal [\textsubbar{l}\textsuperscript{j}] e a aproximante nasal palatal [\~\j], respectivamente. Em outras palavras, o que em Portugal pronuncia-se com consoante, nós pronunciamos com semivogal.

Assim, por exemplo, o que causa o ``estalo'' na palavra ``senha'' [\textprimstress se\~\j a] (cf. [\textprimstress se\textltailn\textturna] no português europeu) é que o ``a'' se diz logo após a semivogal nasalizada [\~\j]: ou seja, o ``estalo'' é residual. E, da mesma maneira, na palavra ``batalha'' [ba\textprimstress ta{\textsubbar{l}\textsuperscript{j}}a], há claramente um ``Quasi-i'' antes do ``a'' (cf. no europeu, [b\textturna\textprimstress ta\textturny\textturna]).

A grafia adotada para esse último fonema no português brasileiro fonético é ``lly'', sendo semelhante às línguas espanhola e francesa, que escrevem a aproximante lateral palatal com o dígrafo ``ll'' (e.g. [exemplo], [exemplo]).

% As razões para essa escolha são duas: 1) como já dito na seção do alfabeto, a letra ``h'' passou a ter o som do ``Erre'' gutural (i.e. a fricativa glotal surda) e, portanto, substituindo o ``r'' no início de palavras e os dois ``r'' intervocálicos, não pode mais ser (mal) utilizada como um ``dígrafo coringa''\footnote{
As razões para essa escolha são duas: 1) como já dito na seção do alfabeto, a letra ``h'' passou a ter o som do ``Erre'' gutural (i.e. a fricativa glotal surda) e, portanto, substituindo o ``r'' no início de palavras e os dois ``r'' intervocálicos, não pode mais ser (mal) utilizada como um ``dígrafo coringa''; 2) porém, mais importantemente, o ``lly'' é, de fato, uma perfeita representação do fonema [\textsubbar{l}\textsuperscript{j}], como explicaremos.

Para entender por que se escolheu o ``lly'' para grafar o [\textsubbar{l}\textsuperscript{j}], convém contrastarmos os sotaques nordestinos com os demais do Brasil.

Tomemos, então, a frase ``Minha filha, hoje é dia de ligar para tua tia.'', que contém quase todas as divergências desse dialeto; e comparemos sua notação fonética com a de outros sotaques brasileiros:
\begin{center}
    \begin{tabular}{l}
        {
            [\textprimstress mĩa
                \textprimstress fi\textsubbar{l}\textsuperscript{j}a,
                \textprimstress\textopeno \textyogh i
                \textepsilon \
                \textprimstress dia
                di
                li\textprimstress gah
                \textprimstress pa\textfishhookr a
                \textprimstress tua
                \textprimstress tia];
        }
        \\
        {
            [\textprimstress mĩa
            \textprimstress fi\textsubbar{l}\textsuperscript{j}a,
            \textprimstress o\textyogh i
            \textepsilon \
            \textprimstress \texttoptiebar{d\textyogh}ia
            \texttoptiebar{d\textyogh}i
            \textsubbar{l}\textsuperscript{j}i\textprimstress gah
            \textprimstress pa\textfishhookr a
            \textprimstress tua
            \textprimstress \texttoptiebar{t\textesh}ia].
        }
    \end{tabular}
\end{center}

Aqui, fica claro que no sotaque nordestino, a pronúncia do ``l'' é consistente e o fonema [\textsubbar{l}\textsuperscript{j}] só é utilizado para o ``lh''. Entretanto, é evidente também que, no resto do Brasil, o ``lh'' tem, na verdade, o mesmo som do ``li'' (cf. ``filha'' [f\textprimstress i\textsubbar{l}\textsuperscript{j}a] e ``família'' [fa\textprimstress mi\textsubbar{l}\textsuperscript{j}a], e compare-se ainda com o italiano, \textit{famiglia} [fa\textprimstress mi\textturny a], onde até escreve-se explicitamente a aproximante lateral palatal com o dígrafo ``gl'').

Os nordestinos, por assim dizer, recitam o silabário corretamente, enquanto os demais trocam o ``li'' por ``lhi'', dizendo: ``la'' [la], ``le'' [le], ``lhi'' [\textsubbar{l}\textsuperscript{j}i], ``lo'' [lo], ``lu'' [lu]. E o mesmo vale para o ``di'' e o ``ti'' (pronunciados ``dji'' [\texttoptiebar{d\textyogh}i] e ``txi'' [\texttoptiebar{t\textesh}i] pela grande maioria dos brasileiros). Com isso, é certo que o ``l'', o ``d'' e o ``t'', seguidos de ``i'' ou ``y'' devem ser novos dígrafos, exceto na variante nordestina da língua brasileira.

Mas, afinal, em que isso nos ajuda com o ``lh''? Ora, eis que o dialeto do nordeste tem, precisamente, a chave para escrevermos o [\textsubbar{l}\textsuperscript{j}] sem ``lh''. A fim de pronunciar esse fonema, basta fazer como se fosse falar um ``li'' ``de baiano'' e interrompê-lo bem no meio com mais um ``li'': o som resultante é exatamente o ``lhi'', incluindo a semivogal [j] residual; porquanto, não é mera convenção que dois ``l'' e um ``i'' formam um ``lh''.

[ni = n\textsuperscript{j}? palatização do ``ni'' = ``nni''?]

O último dígrafo é o ``hr'', ou ``r'' pós-vocálico. Também ausente no português, esse foi adicionado ao idioma brasileiro para expressar todas as maneiras com que pronunciamos o ``r'' depois de vogal (seguido de consoante). E, ao contrário do antigo (mal) uso do ``Agá'', o ``hr'' é um verdadeiro ``dígrafo coringa'', pois pode ser pronunciado como o ``h'' (gutural), o ``r'' (tepe alveolar), ambos ou nenhum.

O primeiro fonema é comum nas regiões litorâneas, desde o Nordeste, ao Rio de Janeiro, chegando até à capital de Santa Catarina. O segundo encontra-se entre os paulistas e sulistas (principalmente gaúchos). E o restante do país utiliza um ``r'' de ``caipira'' [\textturnr], a aproximante alveolar sonora, ou o omite inteiramente.

% Já comentou-se o som do ``h'' e do ``r'' na Seção \ref{pt.section.abc}; e o fonema nulo não requer explicação. Só é preciso explicar, então, como o ``hr'' produz o ``r'' interiorano. De novo, não trata-se de ``convenção'': qual o ``lly'', é por um mecanismo puramente físico que o ``hr'' serve para representar a aproximante alveolar sonora; porque abrindo-se a boca para o ``r'' pós-vocálico gutural, e, logo em seguida, sem pausa alguma, tentar-se realizar o tepe alveolar, a língua chegará ao palato dobrada e não resultará no som de [\textfishhookr], mas de [\textturnr]. Não é necessário um movimento exagerado, apenas rápido o suficiente para que nem o ``h'' nem o ``r'' sejam pronunciados sozinhos. Eis por que o ``hr'' é capaz de ilustrar todos os ``r'' pós-vocálicos do Brasil. 
Já comentou-se o som do ``h'' e do ``r'' na Seção \ref{pt.section.abc}; e o fonema nulo não requer explicação. Só é preciso explicar, então, como o ``hr'' produz o ``r'' interiorano.

De novo, não trata-se de ``convenção'': qual o ``lly'', é por um mecanismo puramente físico que o ``hr'' serve para representar a aproximante alveolar sonora; pois, se abrir-se a boca para fazer o ``r'' pós-vocálico gutural [h], e, logo em seguida, sem pausa alguma, tentar-se realizar o tepe alveolar, a língua chegará ao palato dobrada e não resultará no som de [\textfishhookr], mas de [\textturnr]. Não é necessário um movimento exagerado\footnote{
    Conquanto, se exagerado, obtém-se o retroflexo [\textturnrrtail], também prevalente no interior. 
}, apenas rápido o suficiente para que nem o ``h'' nem o ``r'' sejam pronunciados sozinhos\footnote{
    Nesse sentido, poderia-se dizer até que [\textturnr] é uma consoante composta, especificamente [\texttoptiebar{h\textfishhookr}].
}. Eis por que o ``hr'' é capaz de ilustrar todos os ``r'' pós-vocálicos do Brasil.

\section{Comparação}
\subsection{Análise geral}
[estética da língua brasileira]

[o ``hr'' como símbolo patriótico da união nacional]

\subsubsection{Contagem de caracteres}
[contagem de caracteres para digitar]

[contagem de tempo para digitar]

\subsection{A oração do Santo Rosário}
[]

[Credo]

[Pai Nosso]

[Ave Maria]

[Glória ao Pai]

[Salve Rainha]

[pseudo-conclusão]


Concluímos esse manual com alguns exemplos.

\newpage
\section{Resumo}

\newpage
\section{Referências}

\end{document}