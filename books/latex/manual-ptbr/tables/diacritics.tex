\newcommand{\PtTableDiacritics}{
    \begin{longtblr}[
        caption = {Acentos da língua brasileira},
        label = {pt.table.diacritics}
        ]{
        colspec = {X[0, c, m]X[1, c, m]X[0.5, c, m]},
        width = \linewidth,
        rowhead = 1,
        rowfoot = 0
        }
        Acento             & Nome                 & Exemplo \\
        \toprule
        \textasciiacute    & Acento agudo         & dsds    \\
        \textasciigrave    & Acento grave         & dsds    \\
        \textasciitilde    & Acento nasal         & Princípio    \\
        \textasciicircum   & Acento nasal forte   & Príncipe    \\
        \textasciidieresis & Acento duplo (crase) & Àquela    \\
        \bottomrule
    \end{longtblr}
}
