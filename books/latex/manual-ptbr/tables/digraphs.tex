\newcommand{\PtTableDigraphs}{
    \begin{longtblr}[
        caption = {Dígrafos},
        note{1} = {Dependendo se o ``i" for semivogal ou não.},
        note{2} = {Pós-vocálico, ou ``r'' seguido de vogal e consoante.},
        note{3} = {A palavra ``restaurador'' é interessante, porque contém todos os ``r'' brasileiros: o primeiro é gutural; o segundo, intervocálico; e o terceiro, pós-vocálico.}
        ]{
        colspec = {X[0, c, m]X[0, c, m]X[1, c, m]},
        width = \linewidth,
        rowhead = 1,
        rowfoot = 0
        }
        Antiga Grafia              & Nova Grafia           & IPA                           & Exemplo     \\
        \toprule
        ss                         & s                     & [s]                           &             \\
        sc                         & s                     & [s]                           &             \\
        sç                         & s                     & [s]                           &             \\
        xs                         & s                     & [s]                           &             \\
        xc                         & s                     & [s]                           &             \\
        qu                         & cw                    & [kw]                          & qualidade   \\
        qu                         & c                     & [k]                           & queijo      \\
        gu                         & gw                    & [gw]                          & aguenta     \\
        gu                         & g                     & [g]                           & guerra      \\
        lh                         & lly                   & [?]                           &             \\
        nh                         & \~y, ĩ, î\TblrNote{1} & [\~j]                         &             \\
        ch                         & x                     & [\textesh]                    &             \\
        rr                         & h                     & [h]                           &             \\
        r\TblrNote{2}              & hr                    & [], [], []\TblrNote{3}        & restaurador\TblrNote{3} \\
        di                         & dji                   & [\texttoptiebar{d\textyogh}i] &             \\
        ti                         & txi                   & [\texttoptiebar{t\textesh}i]  &             \\
        li                         & lli, lly\TblrNote{1}  & [?i]                          &             \\
        \bottomrule
    \end{longtblr}
}
