\documentclass[12pt, a4paper]{article}

% \usepackage[backend=biber,style=apa]{biblatex}
\usepackage{bilingualpages}
\usepackage{tipa}
\usepackage{tabularray}
\UseTblrLibrary{booktabs}


% \addbibresource{references.bib}
\newcommand{\PtTableAbc}{
    \begin{longtblr}[
        caption = {Alfabeto português-brasileiro},
        ]{
            colspec = {X[c, m]X[c, m]X[c, m]X[c, m]X[c, m]X[c, m]},
            width = \linewidth,
            rowhead = 1,
            rowfoot = 0
            }
            \toprule
            Aa & Bb & Cc & Dd & Ee & Ff \\
            Gg & Hh & Ii & Jj & Kk & Ll \\
            Mm & Nn & Oo & Pp & Qq & Rr \\
            Ss & Tt & Uu & Vv & Ww & Xx \\
            Yy & Zz &    &    &    &    \\
            \bottomrule
    \end{longtblr}
}

\newcommand{\BrTableAbc}{
    \begin{longtblr}[
        caption = {Awfabétu brazileyru},
        ]{
            colspec = {X[c, m]X[c, m]X[c, m]X[c, m]X[c, m]X[c, m]},
            width = \linewidth,
            rowhead = 1,
            rowfoot = 0
            }
            \toprule
            Aa & Bb & Cc & Dd & Ee & Ff \\
            Gg & Hh & Ii & Yy & Jj & Ll \\
            Mm & Nn & Oo & Pp & Rr & Ss \\
            Tt & Uu & Ww & Vv & Xx & Zz \\
            \bottomrule
    \end{longtblr}
}

% \newcommand{\AbcBrbr}{
%     \begin{longtblr}[
%         caption = {Awfabétu brazileyru},
%         ]{
%             colspec = {X[0, c, m]X[0.5, c, m]X[0.5, c, m]},
%             width = \linewidth,
%             rowhead = 1,
%             rowfoot = 0
%             }
%             \toprule
%             Letra & Nomi & Pronũsya (IPA)\\ 
%             \midrule
%             Aa & A & [-] \\
%             Bb & Be & [-] \\
%             Cc & Ca & [-] \\
%             Dd & De & [-] \\
%             Ee & E & [-] \\
%             Ff & Éfi & [-] \\
%             Gg & Ga & [-] \\
%             Hh & He & [-] \\
%             Ii & I & [-] \\
%             Yy & Cwazi I & [-] \\
%             Jj & Jóta & [-] \\
%             Ll & Élli & [-] \\
%             Mm & Emi & [-] \\
%             Nn & Eni & [-] \\
%             Oo & O & [-] \\
%             Pp & Pe & [-] \\
%             Rr & Éhi & [-] \\
%             Ss & Ési & [-] \\
%             Tt & Te & [-] \\
%             Uu & U & [-] \\
%             Ww & Cwazi U & [-] \\
%             Vv & Ve & [-] \\
%             Xx & Xis & [-] \\
%             Zz & Ze & [-] \\
%             \bottomrule
%     \end{longtblr}    
% }

% \newcommand{\AbcPtBr}{
%     \begin{longtblr}[
%         caption = {Alfabeto português-brasileiro},
%         ]{
%             colspec = {X[0, c, m]X[0.5, c, m]X[0.5, c, m]},
%             width = \linewidth,
%             rowhead = 1,
%             rowfoot = 0
%             }
%             \toprule
%             Letra & Nome & Pronúncia (IPA)\\ 
%             \midrule
%                 Aa & Á & [-] \\% /a/, /ɐ/ \\
%                 Bb & Bê & [-] \\% /b/ \\
%                 Cc & Cê & [-] \\% /k/, /s/ \\
%                 Dd & Dê & [-] \\% /d/ \\
%                 Ee & Ê & [-] \\% /é/, /e/, /ɛ/ \\
%                 Ff & Efe & [-] \\% /f/ \\
%                 Gg & Gê & [-] \\% /g/, /ʒ/ \\
%                 Hh & Agá & [-] \\% /∅/ \\
%                 Ii & I & [-] \\% /i/ \\
%                 Jj & Jota & [-] \\% /ʒ/ \\
%                 Kk & Cá & [-] \\% /k/ \\
%                 Ll & Ele & [-] \\% /l/, /u/ \\
%                 Mm & Eme & [-] \\% /m/, /~/ \\
%                 Nn & Ene & [-] \\% /n/, /~/ \\
%                 Oo & Ô & [-] \\% /ó/, /o/, /ɔ/ \\
%                 Pp & Pê & [-] \\% /p/ \\
%                 Qq & Quê & [-] \\% /k/ \\
%                 Rr & Erre & [-] \\% /ʁ/, /ɾ/, /r/ \\
%                 Ss & Esse & [-] \\% /s/, /z/ \\
%                 Tt & Tê & [-] \\% /t/ \\
%                 Uu & U & [-] \\% /u/ \\
%                 Vv & Vê & [-] \\% /v/ \\
%                 Ww & Dábliu & [-] \\% /u/, /v/ \\
%                 Xx & Xis & [-] \\% /ʃ/, /ks/, /s/, /z/ \\
%                 Yy & Ípsilon & [-] \\% /j/, /i/ \\
%                 Zz & Zê & [-] \\% /z/, /s/ \\
%             \bottomrule
%         \end{longtblr}
% }



\title{Manuaw da llĩgwa brazileyra para luzófonus}
\author{Cao Bittencourt}
\date{\today}

\begin{document}

\maketitle
\thispagestyle{empty}

\newpage \
\thispagestyle{empty}

\newpage
\begin{bilingualpages}
    \rightpage
    \section{Introdução}
    
    \newpage
    \section{Alfabeto}
    Comecemos pelo mais básico, o alfabeto:
    
    \PtTableAbc
    
    Conforme as tabelas, o novo alfabeto brasileiro (à esquerda) tem vinte e quatro letras, enquanto o antigo alfabeto português-brasileiro (à direita) tem vinte e seis. As letras removidas foram o ``k" e o ``q", porque são redundantes. De fato, a primeira delas já era até na antiguidade clássica criticada pelos gramáticos romanos, que achavam-na desnecessária. A letra ``q", por sua vez, foi uma invenção desses mesmos gramáticos para diferenciar o som do ``u" vogal e do ``u" semivogal (cf. as palavras \textit{qui} e \textit{cui}). Nós, no entanto, não temos por que fazer essa distinção, pois a nossa língua tem mais semivogais do que o latim e, além disso, melhores métodos para explicitá-las (ver adiante). Assim sendo, removemos do alfabeto aquela letra, desprezada pelos romanos, e, ironicamente, também essa, que inventaram.

    Não há novas letras no alfabeto, porém muitas das que permaneceram passam a ter novas funções; e, mais importante, uma única função para cada. A letra ``c'', por exemplo, para continuar a discussão acima, tem agora sempre o som de ``k'', nunca de ``s''; na verdade, foi até renomeada para ``Cá'', a fim de deixar isso mais claro. Pelo mesmo motivo, o ``Cê-cedilha'', ``ç'', é substituído por ``s''. E, com isso, acaba-se a ambiguidade entre as consoantes oclusiva velar surda [k] e a fricativa alveolar surda [s].

    Analogamente, a letra ``g'' representa apenas a consoante oclusiva velar sonora [g] e, como o ``c'', foi renomeada para ``Gá'', uma vez que o nome ``Gê'', durante séculos, era pronunciado com a fricativa pós-alveolar sonora [\textyogh] (i.e. o som da letra ``j'' em português). Assim, por exemplo, a palavra ``garagem'', antes escrita com dois ``g'', é, agora explicitamente, ``garaje\~y''.

    Seguindo a ordem alfabética, o antigo ``Agá'', ``h'', deixa de ser uma letra mal utilizada, essencialmente inútil, e passa a ter o som fricativo glotal surdo [h], ou ``Erre" gutural, como é nos demais idiomas da Europa (e.g. nas palavras \textit{home}, \textit{heim} e \textit{hjem}, ou seja, ``lar'' em inglês, alemão e norueguês, respectivamente). Isso significa que o ``r'' é reservado para o tepe alveolar [\textfishhookr] (e.g. em ``para''); e todas as palavras que começavam com ``r'', começam com ``h''; e, pela mesma via, aquelas que tinham dois ``r'', escreve-se também com ``h''. Por fim, remove-se todos os ``h" mudos (e.g. de ``hoje''); e, como nas outras letras, renomeia-se o ``Agá'' para ``Hê'', sinalizando sua nova função.

    Ao contrário das supracitadas línguas germânicas, entretanto, o ``j'' conserva a pronúncia que recebemos dos franceses, não sendo utilizado para o som de ``i'' semivogal (como vimos em \textit{hjem}, acima). Esse som, cujo fonema denota-se por [j], pertence ao ``y'', que, de maneira análoga ao ``h'', antes subutilizado, é agora uma letra muito importante, tendo em vista que o brasileiro é um idioma repleto de semivogais.
    
    Assim, portanto, a indicação das letras semivogais não é nem negligenciada, como vinha sendo desde o Acordo Ortográfico de 1990, tampouco se dá pelo antiquado ``Trema''. Em contraposição, a nova língua brasileira designa letras específicas para esse fim, quais sejam, o ``y'', chamado ``Quase I'', e o ``w'', ou ``Quase U''. Não é necessário mais letras do que essas, porque apenas o ``i'' e o ``u'' são semivogais, enquanto o ``a'', o ``e'' e o ``o'' são sempre vogais (i.e. elas ``quebram'' a sílaba e não aglutinam-se em ditongos e tritongos).
    
    Além disso, o ``y'' e o ``w'' são frequentemente acentuados com o acento nasal, ``\textasciitilde'', e substituem o ``n'' e o ``m'' pós-vocálicos; isso porque uma característica distintiva do brasileiro é que vogais seguidas de ``n'' e ``m'' (com uma consoante depois) sempre produzem um som semivocálico residual, que não é perfeitamente capturado por essas duas consoantes, mas sim por aquelas semivogais nasalizadas (viz. ``\~y'', ``\~w''):

    [exemplos]

    Em praticamente todos os outros idiomas escritos com o alfabeto latino, porém, essa ``semivogal residual'' não acontece, então é correto utilizarem o ``n'' e o ``m'' pós-vocálicos. Mas, como o nosso objetivo é que o \textit{brasileiro} seja consistente, devemos substituí-los por semivogais nasalizadas.

    Finalmente, as duas últimas letras, ``x'' e ``z'' representam, cada, um único som e não se confundem entre si nem com o ``s'', ``c'', etc. Especificamente, o ``x'' tem, agora, sempre o som da fricativa pós-alveolar surda [\textesh] (antigo ``ch''). Já o fonema [z] é grafado pelo ``z'', inclusive nas palavras com ``s'' intervocálico (e.g. ``casa''); e não há mais ``z'' no final de nenhuma palavra. Desse modo, todas as letras no alfabeto têm sua própria função.

    \section{Vogais}

    \subsection{Pronúncia padrão}
    \subsection{Acentos}
    \subsection{Encontros vocálicos}
    \subsection{Lei da gravidade vocálica}
    \subsection{Regras de acentuação}
    % hierarquia dos acentos
    % crase
    
    \section{Encontros Consonantais}
    
    \section{Exemplos}
    
    \leftpage
    \section{Ĩtrodusà\~{w}}
    
    \newpage
    \section{Awfabétu}
    \BrTableAbc

    \subsection{Letras ``nóvas''}
    
    \section{Vogais}
    \subsection{Pronũsya padrà\~w}
    \subsection{Ase\~ytus}
    \subsection{E\~yco\~wtrus vocállicus}
    \subsection{Ley da gravidadji vocállica}
    \subsection{Hégras dji ase\~ytuasà\~w}
    % ierahrcia dus aseỹtus
    % crazi

    \section{E\~yco\~wtrus co\~wsonãtays}

    \section{Eze\~yplus}
    
\end{bilingualpages}

\newpage
\section{Hezumu}

\newpage
\section{Hefere\~ysyas}

\end{document}