\documentclass[12pt, a5paper, titlepage]{article}

\usepackage{pdfpages}
% \usepackage[backend=biber,style=apa]{biblatex}
\usepackage{bilingualpages}
\usepackage{tipa}
\usepackage{tabularray}
\UseTblrLibrary{booktabs}
\usepackage[portuguese]{babel}
\setlength{\emergencystretch}{20pt}


% \addbibresource{references.bib}
\newcommand{\PtTableAbc}{
    \begin{longtblr}[
        caption = {Alfabeto português-brasileiro},
        ]{
            colspec = {X[c, m]X[c, m]X[c, m]X[c, m]X[c, m]X[c, m]},
            width = \linewidth,
            rowhead = 1,
            rowfoot = 0
            }
            \toprule
            Aa & Bb & Cc & Dd & Ee & Ff \\
            Gg & Hh & Ii & Jj & Kk & Ll \\
            Mm & Nn & Oo & Pp & Qq & Rr \\
            Ss & Tt & Uu & Vv & Ww & Xx \\
            Yy & Zz &    &    &    &    \\
            \bottomrule
    \end{longtblr}
}

\newcommand{\BrTableAbc}{
    \begin{longtblr}[
        caption = {Awfabétu brazileyru},
        ]{
            colspec = {X[c, m]X[c, m]X[c, m]X[c, m]X[c, m]X[c, m]},
            width = \linewidth,
            rowhead = 1,
            rowfoot = 0
            }
            \toprule
            Aa & Bb & Cc & Dd & Ee & Ff \\
            Gg & Hh & Ii & Yy & Jj & Ll \\
            Mm & Nn & Oo & Pp & Rr & Ss \\
            Tt & Uu & Ww & Vv & Xx & Zz \\
            \bottomrule
    \end{longtblr}
}

% \newcommand{\AbcBrbr}{
%     \begin{longtblr}[
%         caption = {Awfabétu brazileyru},
%         ]{
%             colspec = {X[0, c, m]X[0.5, c, m]X[0.5, c, m]},
%             width = \linewidth,
%             rowhead = 1,
%             rowfoot = 0
%             }
%             \toprule
%             Letra & Nomi & Pronũsya (IPA)\\ 
%             \midrule
%             Aa & A & [-] \\
%             Bb & Be & [-] \\
%             Cc & Ca & [-] \\
%             Dd & De & [-] \\
%             Ee & E & [-] \\
%             Ff & Éfi & [-] \\
%             Gg & Ga & [-] \\
%             Hh & He & [-] \\
%             Ii & I & [-] \\
%             Yy & Cwazi I & [-] \\
%             Jj & Jóta & [-] \\
%             Ll & Élli & [-] \\
%             Mm & Emi & [-] \\
%             Nn & Eni & [-] \\
%             Oo & O & [-] \\
%             Pp & Pe & [-] \\
%             Rr & Éhi & [-] \\
%             Ss & Ési & [-] \\
%             Tt & Te & [-] \\
%             Uu & U & [-] \\
%             Ww & Cwazi U & [-] \\
%             Vv & Ve & [-] \\
%             Xx & Xis & [-] \\
%             Zz & Ze & [-] \\
%             \bottomrule
%     \end{longtblr}    
% }

% \newcommand{\AbcPtBr}{
%     \begin{longtblr}[
%         caption = {Alfabeto português-brasileiro},
%         ]{
%             colspec = {X[0, c, m]X[0.5, c, m]X[0.5, c, m]},
%             width = \linewidth,
%             rowhead = 1,
%             rowfoot = 0
%             }
%             \toprule
%             Letra & Nome & Pronúncia (IPA)\\ 
%             \midrule
%                 Aa & Á & [-] \\% /a/, /ɐ/ \\
%                 Bb & Bê & [-] \\% /b/ \\
%                 Cc & Cê & [-] \\% /k/, /s/ \\
%                 Dd & Dê & [-] \\% /d/ \\
%                 Ee & Ê & [-] \\% /é/, /e/, /ɛ/ \\
%                 Ff & Efe & [-] \\% /f/ \\
%                 Gg & Gê & [-] \\% /g/, /ʒ/ \\
%                 Hh & Agá & [-] \\% /∅/ \\
%                 Ii & I & [-] \\% /i/ \\
%                 Jj & Jota & [-] \\% /ʒ/ \\
%                 Kk & Cá & [-] \\% /k/ \\
%                 Ll & Ele & [-] \\% /l/, /u/ \\
%                 Mm & Eme & [-] \\% /m/, /~/ \\
%                 Nn & Ene & [-] \\% /n/, /~/ \\
%                 Oo & Ô & [-] \\% /ó/, /o/, /ɔ/ \\
%                 Pp & Pê & [-] \\% /p/ \\
%                 Qq & Quê & [-] \\% /k/ \\
%                 Rr & Erre & [-] \\% /ʁ/, /ɾ/, /r/ \\
%                 Ss & Esse & [-] \\% /s/, /z/ \\
%                 Tt & Tê & [-] \\% /t/ \\
%                 Uu & U & [-] \\% /u/ \\
%                 Vv & Vê & [-] \\% /v/ \\
%                 Ww & Dábliu & [-] \\% /u/, /v/ \\
%                 Xx & Xis & [-] \\% /ʃ/, /ks/, /s/, /z/ \\
%                 Yy & Ípsilon & [-] \\% /j/, /i/ \\
%                 Zz & Zê & [-] \\% /z/, /s/ \\
%             \bottomrule
%         \end{longtblr}
% }


\newcommand{\PtTableVowels}{
    \begin{longtblr}[
        caption = {Pronúncia padrão},
        ]{
        colspec = {X[0, c, m]X[1, c, m]X[0.5, c, m]},
        width = \linewidth,
        rowhead = 1,
        rowfoot = 0
        }
        Letra & Pronúncia padrão & Exemplo \\ 
        \toprule
        Aa    & Agudo            & dsds \\  
        Ee    & Grave            & dsds \\  
        Ii    & Agudo            & dsds \\  
        Oo    & Grave            & dsds \\  
        Uu    & Agudo            & dsds \\  
        Yy    & Agudo            & dsds \\  
        Ww    & Agudo            & dsds \\  
        \bottomrule
    \end{longtblr}
}

\newcommand{\BrTableVowels}{
    \begin{longtblr}[
        caption = {Pronũsya padrà\~w},
        ]{
        colspec = {X[0, c, m]X[1, c, m]X[0.5, c, m]},
        width = \linewidth,
        rowhead = 1,
        rowfoot = 0
        }
        Letra & Pronũsya padrà\~w & Eze\~yplu \\
        \toprule
        Aa    & Agudu             & dsds      \\ 
        Ee    & Gravi             & dsds      \\ 
        Ii    & Agudu             & dsds      \\ 
        Oo    & Gravi             & dsds      \\ 
        Uu    & Agudu             & dsds      \\ 
        Yy    & Agudu             & dsds      \\ 
        Ww    & Agudu             & dsds      \\ 
        \bottomrule
    \end{longtblr}
}

\newcommand{\PtTableDiacritics}{
    \begin{longtblr}[
        caption = {Acentos da língua brasileira},
        label = {pt.table.diacritics}
        ]{
        colspec = {X[0, c, m]X[1, c, m]X[0.5, c, m]},
        width = \linewidth,
        rowhead = 1,
        rowfoot = 0
        }
        Acento             & Nome                 & Exemplo \\
        \toprule
        \textasciiacute    & Acento agudo         & dsds    \\
        \textasciigrave    & Acento grave         & dsds    \\
        \textasciitilde    & Acento nasal         & dsds    \\
        \textasciicircum   & Acento nasal forte   & dsds    \\
        \textasciidieresis & Acento duplo (crase) & dsds    \\
        \bottomrule
    \end{longtblr}
}

\newcommand{\BrTableDiacritics}{
    \begin{longtblr}[
        caption = {Ase\~ytus da llĩgwa brazileyra},
        label = {br.table.diacritics}
        ]{
        colspec = {X[0, c, m]X[1, c, m]X[0.5, c, m]},
        width = \linewidth,
        rowhead = 1,
        rowfoot = 0
        }
        Ase\~ytu           & Nomi                   & Eze\~yplu \\
        \toprule
        \textasciiacute    & Ase\~ytu agudu         & dsds      \\
        \textasciigrave    & Ase\~ytu gravi         & dsds      \\
        \textasciitilde    & Ase\~ytu nazaw         & dsds      \\
        \textasciicircum   & Ase\~ytu nazaw fóhrtxi & dsds      \\
        \textasciidieresis & Ase\~ytu duplu (crazi) & dsds      \\
        \bottomrule
    \end{longtblr}
}

\title{Manuaw da llĩgwa brazileyra para luzófonus}
\author{Cao Bittencourt}
\date{}

\begin{document}

\maketitle
\thispagestyle{empty}

\newpage \
\thispagestyle{empty}

\newpage
\begin{bilingualpages}
    \rightpage
    \section{Introdução}
    
    \newpage
    \section{Alfabeto}
    Comecemos pelo mais básico, o alfabeto:
    
    \PtTableAbc
    Conforme as tabelas, o novo alfabeto brasileiro (à esquerda) tem vinte e quatro letras, enquanto o antigo alfabeto português-brasileiro (à direita) tem vinte e seis. As letras removidas foram o ``k" e o ``q", porque são redundantes. De fato, a primeira delas já era até na antiguidade clássica criticada pelos gramáticos romanos, que achavam-na desnecessária. A letra ``q", por sua vez, foi uma invenção desses mesmos gramáticos para diferenciar o som do ``u" vogal e do ``u" semivogal (cf. as palavras \textit{qui} e \textit{cui}). Nós, no entanto, não temos por que fazer essa distinção, pois a nossa língua tem mais semivogais do que o latim e, além disso, melhores métodos para explicitá-las (ver adiante). Assim sendo, removemos do alfabeto aquela letra, desprezada pelos romanos, e, ironicamente, também essa, que inventaram.

    Não há novas letras no alfabeto, porém muitas das que permaneceram passam a ter novas funções; e, mais importante, uma única função para cada. A letra ``c'', por exemplo, para continuar a discussão acima, tem agora sempre o som de ``k'', nunca de ``s''; na verdade, foi até renomeada para ``Cá'' [ka], a fim de deixar isso mais claro. Pelo mesmo motivo, o ``Cê-cedilha'', ``ç'', é substituído por ``s''. E, com isso, acaba-se a ambiguidade entre as consoantes oclusiva velar surda [k] e a fricativa alveolar surda [s].

    Analogamente, a letra ``g'' representa apenas a consoante oclusiva velar sonora [g] e, como o ``c'', foi renomeada para ``Gá'' [ga], uma vez que o nome ``Gê'' [\textyogh e], durante séculos, era pronunciado com a fricativa pós-alveolar sonora [\textyogh] (i.e. o som da letra ``j'' em português). Assim, por exemplo, a palavra ``garagem'', antes escrita com dois ``g'', é, agora explicitamente, ``garáje\~y''.

    Seguindo a ordem alfabética, o antigo ``Agá'', ``h'', deixa de ser uma letra mal utilizada, essencialmente inútil, e passa a ter o som fricativo glotal surdo [h], ou ``Erre" gutural, como é nos demais idiomas da Europa (e.g. nas palavras \textit{home}, \textit{heim} e \textit{hjem}, ou seja, ``lar'' em inglês, alemão e norueguês, respectivamente). Isso significa que o ``r'' é reservado para o tepe alveolar [\textfishhookr] (e.g. em ``para''); e todas as palavras que começavam com ``r'', começam com ``h''; e, pela mesma via, aquelas que tinham dois ``r'', escreve-se também com ``h''. Por fim, remove-se todos os ``h" mudos (e.g. ``hoje''); e, como nas outras letras, renomeia-se o ``Agá'' para ``Erre'' [\textepsilon hi] e o ``Erre'' para ``Eri'' [\textepsilon \textfishhookr i], sinalizando suas novas funções.

    Ao contrário das supracitadas línguas germânicas, entretanto, o ``j'' conserva a pronúncia que recebemos dos franceses, não sendo utilizado para o som de ``i'' semivogal (como vimos em \textit{hjem}, acima). Esse som, cujo fonema denota-se por [j], pertence ao ``y'', que, de maneira análoga ao ``h'', antes subutilizado, é agora uma letra muito importante, tendo em vista que o brasileiro é um idioma repleto de semivogais.
    
    Assim, portanto, a indicação das letras semivogais não é nem negligenciada, como vinha sendo desde o Acordo Ortográfico de 1990, tampouco se dá pelo antiquado ``Trema''. Em contraposição, a nova língua brasileira designa letras específicas para esse fim, quais sejam, o ``y'', chamado ``Quasi-i'', e o ``w'', ou ``Quasi-u''. Não é necessário mais letras do que essas, porque apenas o ``i'' e o ``u'' são semivogais, enquanto o ``a'', o ``e'' e o ``o'' são sempre vogais (i.e. elas ``quebram'' a sílaba e não aglutinam-se em ditongos e tritongos).
    
    Além disso, o ``y'' e o ``w'' são frequentemente acentuados com o acento nasal, ``\textasciitilde'', e substituem o ``n'' e o ``m'' pós-vocálicos; isso porque uma característica distintiva do brasileiro é que vogais seguidas de ``n'' e ``m'' (com uma consoante depois) sempre produzem um som semivocálico residual, que não é perfeitamente capturado por essas duas consoantes, mas sim por aquelas semivogais nasalizadas (viz. ``\~y'', ``\~w''):

    [exemplos]

    Em praticamente todos os outros idiomas escritos com o alfabeto latino, porém, essa ``semivogal residual'' não acontece, então é correto utilizarem o ``n'' e o ``m'' pós-vocálicos. Mas, como o nosso objetivo é que o \textit{brasileiro} seja consistente, devemos substituí-los por semivogais nasalizadas.

    Finalmente, as duas últimas letras, ``x'' e ``z'' representam, cada, um único som e não se confundem entre si nem com o ``s'', ``c'', etc. Especificamente, o ``x'' tem, agora, sempre o som da fricativa pós-alveolar surda [\textesh] (antigo ``ch''). Já o fonema [z] é grafado pelo ``z'', inclusive nas palavras com ``s'' intervocálico (e.g. ``casa''); e não há mais ``z'' no final de nenhuma palavra. Desse modo, todas as letras no alfabeto têm sua própria função.

    \section{Vogais}
    Como aludido acima, as vogais na língua brasileira são ``a'', ``e'', ``i'', ``o'', ``u''; e as semivogais, ``y'' e ``w'' (``Quasi-i'' e ``Quasi-u''). As vogais formam hiátos se adjacentes, mas as semivogais aglutinam-se.
    
    Por exemplo,

    [exemplos]

    Ademais, porque visamos a consistência fonética (i.e. que se escreva como se diz), precisamos distinguir não só entre vogais e semivogais, mas ainda entre as agudas, graves e nasais. Denotá-las explicitamente exigiria ou uma letra para cada som (como é no Alfabeto Fonético Internacional) ou algum sistema de acentuação. A primeira opção não seria nem um pouco prática; a segunda, no entanto, também pode tornar-se trabalhosa se não implementada direito.
    
    Em particular, para evitar excessivos acentos, devemos convencionar uma ``pronúncia padrão'' para cada vogal (viz. a mais frequente), e indicar com acentos apenas quando a pronúncia for diferente.

    A tabela abaixo define a pronúncia padrão das vogais e semivogais brasileiras:

    \PtTableVowels

    Como pode-se perceber, os fonemas vocálicos são os mesmos do português tradicional. Então, nesse sentido, excetuando a adição das semivogais, não há nada de novo. As pronúncias alternativas, porém, não são as mesmas, conquanto sejam mais acuradas do que no português. Expliquemo-las na seguinte seção.

    \subsection{Acentos}
    Para entender as pronúncias alternativas das vogais e semivogais brasileiras, convém definirmos, primeiro, os acentos que as indicam:

    \PtTableDiacritics
    
    função dos acentos: 1) explicitar pronúncia quando não é a pronúncia padrão; 2) indicar sílaba tônica quando não é autoevidente; 3) diferenciar palavras.

    lista e hierarquia dos acentos.

    \subsection{Encontros vocálicos}
    vogal + y
    
    vogal + w
    
    y + vogal
    
    w + vogal
    
    y + vogal + w
    
    y + vogal + y

    w + vogal + w
    
    w + vogal + y
    
    
    \subsection{Lei da gravidade vocálica}
    e $\rightarrow$ i $\rightarrow$ y

    o $\rightarrow$ u $\rightarrow$ w
    
    \subsection{Regras de acentuação}
    % hierarquia dos acentos
    % crase
    
    \section{Dígrafos}
    nh $\rightarrow$ \~y, ĩ, î
    
    lh $\rightarrow$ lly, lli
    
    ss $\rightarrow$ s
    
    sc $\rightarrow$ s
    
    sç $\rightarrow$ s
    
    xs $\rightarrow$ s
    
    xc $\rightarrow$ s
    
    ch $\rightarrow$ x
    
    rr $\rightarrow$ h

    qu $\rightarrow$ cw (e.g. qualidade), c (queijo)
    
    gu $\rightarrow$ gw (e.g. guitarra), g (guerra)

    r pós-vocálico (seguido de consoante) $\rightarrow$ hr

    di $\rightarrow$ dji
    
    ti $\rightarrow$ txi

    li $\rightarrow$ lli, lly

    Ademais, todos os dígrafos vocálicos (viz. vogal seguida de ``n'' ou ``m'') foram substituídos por vogais nasalizadas (seguidas de ``\~y'' ou ``\~w'' quando resultam em semivogal residual), como explicado no capítulo anterior.
    
    \section{Exemplos}
    
    \leftpage
    \section{Ĩtrodusà\~w}
    
    \newpage
    \section{Awfabétu}
    Comesemus pelu mays bázicu, u awfabétu:
    \BrTableAbc
    
    Co\~wfóhrmi as tabélas, u novu awfabétu brazileyru (ä escehrda) te\~y vĩtxi i cwatru letras, ĩcwãtu u ãtxigu awfabétu pohrtugeys-brazileyru (ä djireyta) te\~y vĩtxi i seys. As letras hemovidas fòrà\~w u ``k'' i u ``q'', pohrcè sà\~w hedũdãtxis. Dji fatu, a primeyra délas ja éra até na ãtxigwidadji clásica critxicada pelus gramátxicus homànus, ci axávà\~w-na desnesesarya. A letra ``q'', pohr sua veys, foy uma ĩve\~ysà\~w desis mesmus gramátxicus para djifere\~ysiahr u so\~w du ``u'' vogaw i du ``u'' semivogaw (cf. as palavras \textit{qui} i \textit{cui}). Nós, nu e\~ytãtu, nà\~w temus pohr ce fazehr ésa djistĩsà\~w, poys a nósa llĩgwa te\~y mays semivogays du ci u latî i, ale\~y djisu, mellyóris métodus para espllisitá-las (vehr adjiãtxi). Asî se\~ydu, hemovemus du awfabétu acéla letra, desprezada pelus homànus, i, ironicame\~ytxi tãbe\~y ésa, ci ĩve\~ytárà\~w.

    Nà\~w á nóvas letras nu awfabétu, pore\~y mu\~ytas das ci pehrmanesèrà\~w pásà\~w a tehr nóvas fũso\~ys; i, mays ĩpohrtãtxi, uma única fũsà\~w para cada. A letra ``c'', pohr eze\~yplu, para co\~wtxinuahr a djiscusà\~w asima, te\~y agóra se\~ypri u so\~w dji ``k'', nũca dji ``s''; na vehrdadji, foy até henomiada para ``Ca'' [ka], a fĩ dji deyxahr isu mays claru. Pelu mesmu motxivu, u ``Se-sidjillya'', ``ç'', é subistxituidu pohr ``s''. I, co\~w isu, acaba-si a ãbigwidadji e\~ytri as co\~wsoãtxis ocluziva velahr suhrda [k] i a fricatxiva awveolahr suhrda [s].

    Analogame\~ytxi, a letra ``g'' hepreze\~yta apenas a co\~wsoãtxi ocluziva velahr sonóra [g] i, comu u ``c'', foy henomiada para ``Ga'' [ga], uma veys ci u nomi ``Je'' [\textyogh e], durãtxi séculus, éra pronũsiadu co\~w a fricatxiva pós-awveolahr sonóra [\textyogh] (i.e. o so\~w da letra ``j'' ĩ pohrtugeys). Asî, pohr eze\~yplu, a palavra ``garagem'', ãtxis iscrita co\~w doys ``g'', é, agóra espllisitame\~ytxi, ``garáje\~y''.

    Segĩdu a óhrde\~y awfabétxica, u ãtxigu ``Agá'', ``h'', deyxa dji sehr uma letra maw utxillizada, ese\~ysiawme\~ytxi inútxiw, i pasa a tehr u so\~w fricatxivu glotaw suhrdu [h], ow ``Éhi'' guturaw, comu é nus demays idjiomas da Ewrópa (e.g. nas palavras \textit{home}, \textit{heim} i \textit{hjem}, ow seja, ``lahr'' ĩ Ĩgleys, Alemà\~w i Noruegeys, hespectxivame\~ytxi). Isu significa ci u ``r'' é hezehrvadu para u tépi awveolahr [\textfishhookr] (e.g. ĩ ``para''); i todas as palavras ci comesávà\~w co\~w ``r'', comésà\~w co\~w ``h''; i, pela mesma via, acélas ci txîà\~w doys ``r'', iscrévi-si tãbe\~y co\~w ``h''. Pohr fĩ, hemóvi-si todus us ``h'' mudus (e.g. ``oji''); i, comu nas owtras letras, henomeya-si u ``Agá'' para ``Éhi'' [\textepsilon hi] i u ``Éhi'' para ``Éri'' [\textepsilon \textfishhookr i], sinalizãdu suas nóvas fũso\~ys.

    Aw co\~wtraryu das suprasitadas llĩgwas jehrmànicas, e\~ytretãtu, u ``j'' co\~wséhrva a pronũsya ci hesebemus dus frãsezis, nà\~w se\~ydu utxillizadu para u so\~w dji ``i'' semivogaw (comu vimus ĩ \textit{hjem}, asima). Esi so\~w, cuju fonema denóta-si pohr [j] pehrte\~ysi aw ``y'', ci, dji maneyra análoga aw ``h'', ãtxis sub'utxillizadu, é agóra uma letra mu\~ytu ĩpohrtãtxi, te\~ydu ĩ vista ci u Brazileyru é ũ idjioma replétu dji semivogays. 

    Asî, pohrtãtu, a ĩdjicasà\~w das letras semivogays nà\~w é ne\~y negllije\~ysiada, comu vîa se\~ydu deysdji u Acohrdu Ohrtográficu dji 1990, tà\~wpòwcu si dá pelu ãtxicwadu ``Trema''. Ĩ co\~wtrapozisà\~w, a nóva llĩgwa brazileyra dezigna letras espesíficas para esi fĩ, cways sèjà\~w, u ``y'', xamadu ``Cwazi-i'', i u ``w'', ow ``Cwazi-u''. Nà\~w é nesesaryu mays letras du ci ésas, pohrcè apenas u ``i'' i u ``u'' sà\~w semivogays, ĩcwãtu u ``a'', u ``e'' i u ``o'' sà\~w se\~ypri vogays (i.e. élas ``cébrà\~w'' a sílaba i nà\~w aglutxínà\~w-si ĩ djito\~wgus i trito\~wgus).

    Ale\~y djisu, u ``y'' i u ``w'' sà\~w frecwe\~ytxime\~ytxi ase\~ytuadus co\~w u ase\~ytu nazaw, ``\textasciitilde'', i substxitúe\~y u ``n'' i u ``m'' pós-vocállicus; isu pohrcè uma característxica djistĩtxiva du Brazileyru é ci vogays segidas dji ``n'' i ``m'' (co\~w uma co\~wsoãtxi depoys) se\~ypri prodúze\~y ũ so\~w semivocállicu heziduaw, ci nà\~w é pehrfeytame\~ytxi capturadu pohr ésas duas co\~wsoãtxis, mas sĩ pohr acélas semivogays nazallizadas (viz. ``\~y'', ``\~w''):

    [eze\~yplus]

    Em praticamente todos os outros idiomas escritos com o alfabeto latino, porém, essa ``semivogal residual'' não acontece, então é correto utilizarem o ``n'' e o ``m'' pós-vocálicos. Mas, como o nosso objetivo é que o \textit{brasileiro} seja consistente, devemos substituí-los por semivogais nasalizadas.

    Finalmente, as duas últimas letras, ``x'' e ``z'' representam, cada, um único som e não se confundem entre si nem com o ``s'', ``c'', etc. Especificamente, o ``x'' tem, agora, sempre o som da fricativa pós-alveolar surda [\textesh] (antigo ``ch''). Já o fonema [z] é grafado pelo ``z'', inclusive nas palavras com ``s'' intervocálico (e.g. ``casa''); e não há mais ``z'' no final de nenhuma palavra. Desse modo, todas as letras no alfabeto têm sua própria função.

    \section{Vogais}
    \subsection{Pronũsya padrà\~w}
    \subsection{Ase\~ytus}
    \subsection{E\~yco\~wtrus vocállicus}
    \subsection{Ley da gravidadji vocállica}
    \subsection{Hégras dji ase\~ytuasà\~w}
    % ierahrcia dus aseỹtus
    % crazi

    \section{E\~yco\~wtrus co\~wsonãtays}

    \section{Eze\~yplus}
    
\end{bilingualpages}

\newpage
\section{Hezumu}

\newpage
\section{Hefere\~ysyas}

\end{document}