\documentclass[12pt, a5paper, titlepage]{article}

\usepackage{pdfpages}
% \usepackage[backend=biber,style=apa]{biblatex}
\usepackage{amsmath}
\usepackage{bilingualpages}
\usepackage{tipa}
\usepackage{tabularray}
\UseTblrLibrary{booktabs}
\usepackage[portuguese]{babel}
\setlength{\emergencystretch}{20pt}


% \addbibresource{references.bib}
\newcommand{\PtTableAbc}{
    \begin{longtblr}[
        caption = {Alfabeto português-brasileiro},
        ]{
            colspec = {X[c, m]X[c, m]X[c, m]X[c, m]X[c, m]X[c, m]},
            width = \linewidth,
            rowhead = 1,
            rowfoot = 0
            }
            \toprule
            Aa & Bb & Cc & Dd & Ee & Ff \\
            Gg & Hh & Ii & Jj & Kk & Ll \\
            Mm & Nn & Oo & Pp & Qq & Rr \\
            Ss & Tt & Uu & Vv & Ww & Xx \\
            Yy & Zz &    &    &    &    \\
            \bottomrule
    \end{longtblr}
}

\newcommand{\BrTableAbc}{
    \begin{longtblr}[
        caption = {Awfabétu brazileyru},
        ]{
            colspec = {X[c, m]X[c, m]X[c, m]X[c, m]X[c, m]X[c, m]},
            width = \linewidth,
            rowhead = 1,
            rowfoot = 0
            }
            \toprule
            Aa & Bb & Cc & Dd & Ee & Ff \\
            Gg & Hh & Ii & Yy & Jj & Ll \\
            Mm & Nn & Oo & Pp & Rr & Ss \\
            Tt & Uu & Ww & Vv & Xx & Zz \\
            \bottomrule
    \end{longtblr}
}

% \newcommand{\AbcBrbr}{
%     \begin{longtblr}[
%         caption = {Awfabétu brazileyru},
%         ]{
%             colspec = {X[0, c, m]X[0.5, c, m]X[0.5, c, m]},
%             width = \linewidth,
%             rowhead = 1,
%             rowfoot = 0
%             }
%             \toprule
%             Letra & Nomi & Pronũsya (IPA)\\ 
%             \midrule
%             Aa & A & [-] \\
%             Bb & Be & [-] \\
%             Cc & Ca & [-] \\
%             Dd & De & [-] \\
%             Ee & E & [-] \\
%             Ff & Éfi & [-] \\
%             Gg & Ga & [-] \\
%             Hh & He & [-] \\
%             Ii & I & [-] \\
%             Yy & Cwazi I & [-] \\
%             Jj & Jóta & [-] \\
%             Ll & Élli & [-] \\
%             Mm & Emi & [-] \\
%             Nn & Eni & [-] \\
%             Oo & O & [-] \\
%             Pp & Pe & [-] \\
%             Rr & Éhi & [-] \\
%             Ss & Ési & [-] \\
%             Tt & Te & [-] \\
%             Uu & U & [-] \\
%             Ww & Cwazi U & [-] \\
%             Vv & Ve & [-] \\
%             Xx & Xis & [-] \\
%             Zz & Ze & [-] \\
%             \bottomrule
%     \end{longtblr}    
% }

% \newcommand{\AbcPtBr}{
%     \begin{longtblr}[
%         caption = {Alfabeto português-brasileiro},
%         ]{
%             colspec = {X[0, c, m]X[0.5, c, m]X[0.5, c, m]},
%             width = \linewidth,
%             rowhead = 1,
%             rowfoot = 0
%             }
%             \toprule
%             Letra & Nome & Pronúncia (IPA)\\ 
%             \midrule
%                 Aa & Á & [-] \\% /a/, /ɐ/ \\
%                 Bb & Bê & [-] \\% /b/ \\
%                 Cc & Cê & [-] \\% /k/, /s/ \\
%                 Dd & Dê & [-] \\% /d/ \\
%                 Ee & Ê & [-] \\% /é/, /e/, /ɛ/ \\
%                 Ff & Efe & [-] \\% /f/ \\
%                 Gg & Gê & [-] \\% /g/, /ʒ/ \\
%                 Hh & Agá & [-] \\% /∅/ \\
%                 Ii & I & [-] \\% /i/ \\
%                 Jj & Jota & [-] \\% /ʒ/ \\
%                 Kk & Cá & [-] \\% /k/ \\
%                 Ll & Ele & [-] \\% /l/, /u/ \\
%                 Mm & Eme & [-] \\% /m/, /~/ \\
%                 Nn & Ene & [-] \\% /n/, /~/ \\
%                 Oo & Ô & [-] \\% /ó/, /o/, /ɔ/ \\
%                 Pp & Pê & [-] \\% /p/ \\
%                 Qq & Quê & [-] \\% /k/ \\
%                 Rr & Erre & [-] \\% /ʁ/, /ɾ/, /r/ \\
%                 Ss & Esse & [-] \\% /s/, /z/ \\
%                 Tt & Tê & [-] \\% /t/ \\
%                 Uu & U & [-] \\% /u/ \\
%                 Vv & Vê & [-] \\% /v/ \\
%                 Ww & Dábliu & [-] \\% /u/, /v/ \\
%                 Xx & Xis & [-] \\% /ʃ/, /ks/, /s/, /z/ \\
%                 Yy & Ípsilon & [-] \\% /j/, /i/ \\
%                 Zz & Zê & [-] \\% /z/, /s/ \\
%             \bottomrule
%         \end{longtblr}
% }


\newcommand{\PtTableVowels}{
    \begin{longtblr}[
        caption = {Pronúncia padrão},
        ]{
        colspec = {X[0, c, m]X[1, c, m]X[0.5, c, m]},
        width = \linewidth,
        rowhead = 1,
        rowfoot = 0
        }
        Letra & Pronúncia padrão & Exemplo \\ 
        \toprule
        Aa    & Agudo            & dsds \\  
        Ee    & Grave            & dsds \\  
        Ii    & Agudo            & dsds \\  
        Oo    & Grave            & dsds \\  
        Uu    & Agudo            & dsds \\  
        Yy    & Agudo            & dsds \\  
        Ww    & Agudo            & dsds \\  
        \bottomrule
    \end{longtblr}
}

\newcommand{\BrTableVowels}{
    \begin{longtblr}[
        caption = {Pronũsya padrà\~w},
        ]{
        colspec = {X[0, c, m]X[1, c, m]X[0.5, c, m]},
        width = \linewidth,
        rowhead = 1,
        rowfoot = 0
        }
        Letra & Pronũsya padrà\~w & Eze\~yplu \\
        \toprule
        Aa    & Agudu             & dsds      \\ 
        Ee    & Gravi             & dsds      \\ 
        Ii    & Agudu             & dsds      \\ 
        Oo    & Gravi             & dsds      \\ 
        Uu    & Agudu             & dsds      \\ 
        Yy    & Agudu             & dsds      \\ 
        Ww    & Agudu             & dsds      \\ 
        \bottomrule
    \end{longtblr}
}

\newcommand{\PtTableDiacritics}{
    \begin{longtblr}[
        caption = {Acentos da língua brasileira},
        label = {pt.table.diacritics}
        ]{
        colspec = {X[0, c, m]X[1, c, m]X[0.5, c, m]},
        width = \linewidth,
        rowhead = 1,
        rowfoot = 0
        }
        Acento             & Nome                 & Exemplo \\
        \toprule
        \textasciiacute    & Acento agudo         & dsds    \\
        \textasciigrave    & Acento grave         & dsds    \\
        \textasciitilde    & Acento nasal         & dsds    \\
        \textasciicircum   & Acento nasal forte   & dsds    \\
        \textasciidieresis & Acento duplo (crase) & dsds    \\
        \bottomrule
    \end{longtblr}
}

\newcommand{\BrTableDiacritics}{
    \begin{longtblr}[
        caption = {Ase\~ytus da llĩgwa brazileyra},
        label = {br.table.diacritics}
        ]{
        colspec = {X[0, c, m]X[1, c, m]X[0.5, c, m]},
        width = \linewidth,
        rowhead = 1,
        rowfoot = 0
        }
        Ase\~ytu           & Nomi                   & Eze\~yplu \\
        \toprule
        \textasciiacute    & Ase\~ytu agudu         & dsds      \\
        \textasciigrave    & Ase\~ytu gravi         & dsds      \\
        \textasciitilde    & Ase\~ytu nazaw         & dsds      \\
        \textasciicircum   & Ase\~ytu nazaw fóhrtxi & dsds      \\
        \textasciidieresis & Ase\~ytu duplu (crazi) & dsds      \\
        \bottomrule
    \end{longtblr}
}
\newcommand{\PtTableDigraphs}{
    \begin{longtblr}[
        caption = {Dígrafos},
        note{1} = {Dependendo se o ``i" for semivogal ou não.},
        note{2} = {Isto é, o ``r'' seguido de vogal e consoante.}
        ]{
        colspec = {X[0, c, m]X[0, c, m]X[1, c, m]},
        width = \linewidth,
        rowhead = 1,
        rowfoot = 0
        }
        Antiga Grafia              & Nova Grafia           & IPA                           & Exemplo     \\
        \toprule
        ss                         & s                     & [s]                           &             \\
        sc                         & s                     & [s]                           &             \\
        sç                         & s                     & [s]                           &             \\
        xs                         & s                     & [s]                           &             \\
        xc                         & s                     & [s]                           &             \\
        qu                         & cw                    & [kw]                          & qualidade   \\
        qu                         & c                     & [k]                           & queijo      \\
        gu                         & gw                    & [gw]                          & aguenta     \\
        gu                         & g                     & [g]                           & guerra      \\
        lh                         & lly, lli\TblrNote{1}  & [?]                           &             \\
        nh                         & \~y, ĩ, î\TblrNote{1} & [\~j]                         &             \\
        ch                         & x                     & [\textesh]                    &             \\
        rr                         & h                     & [h]                           &             \\
        r pós-vocálico\TblrNote{2} & hr                    & [], [], []\TblrNote{3}        & restaurador \\
        di                         & dji                   & [\texttoptiebar{d\textyogh}i] &             \\
        ti                         & txi                   & [\texttoptiebar{t\textesh}i]  &             \\
        li                         & lli, lly              & [?i]                          &             \\
        \bottomrule
    \end{longtblr}
}

\newcommand{\BrTableDigraphs}{
    \begin{longtblr}[
        caption = {Djígrafus},
        note{1} = {Depe\~yde\~ydu si u ``i'' fohr semivogaw ow na\~w.},
        note{2} = {Istu é, u ``r'' segidu dji vogaw i co\~wsoãtxi.}
        ]{
        colspec = {X[0, c, m]X[0, c, m]X[1, c, m]},
        width = \linewidth,
        rowhead = 1,
        rowfoot = 0
        }
        Ãtxiga Grafia               & Nóva Grafia           & Eze\~yplu          \\
        \toprule
        nh                          & \~y, ĩ, î\TblrNote{1} &                    \\
        lh                          & lly, lli\TblrNote{1}  &                    \\
        ss                          & s                     &                    \\
        sc                          & s                     &                    \\
        sç                          & s                     &                    \\
        xs                          & s                     &                    \\
        xc                          & s                     &                    \\
        ch                          & x                     &                    \\
        rr                          & h                     &                    \\
        qu                          & cw, c                 & cwallidadji, ceyju \\
        gu                          & gw, g                 & agwe\~yta, géha    \\
        r pós-vocállicu\TblrNote{2} & hr                    &                    \\
        di                          & dji                   &                    \\
        ti                          & txi                   &                    \\
        li                          & lli, lly              &                    \\
        \bottomrule
    \end{longtblr}
}


\title{Manuaw da llĩgwa brazileyra para luzófonus}
\author{Cao Bittencourt}
\date{}

\begin{document}

\maketitle
\thispagestyle{empty}

\newpage \
\thispagestyle{empty}

\newpage
\begin{bilingualpages}
    \rightpage
    \section{Introdução}

    \newpage
    \section{Alfabeto}
    Comecemos pelo mais básico, o alfabeto:

    \PtTableAbc
    Conforme as tabelas, o novo alfabeto brasileiro (à esquerda) tem vinte e quatro letras, enquanto o antigo alfabeto português-brasileiro (à direita) tem vinte e seis. As letras removidas foram o ``k" e o ``q", porque são redundantes. De fato, a primeira delas já era até na antiguidade clássica criticada pelos gramáticos romanos, que achavam-na desnecessária. A letra ``q", por sua vez, foi uma invenção desses mesmos gramáticos para diferenciar o som do ``u" vogal e do ``u" semivogal (cf. as palavras \textit{qui} e \textit{cui}). Nós, no entanto, não temos por que fazer essa distinção, pois a nossa língua tem mais semivogais do que o latim e, além disso, melhores métodos para explicitá-las (ver adiante). Assim sendo, removemos do alfabeto aquela letra, desprezada pelos romanos, e, ironicamente, também essa, que inventaram.

    Não há novas letras no alfabeto, porém muitas das que permaneceram passam a ter
    novas funções; e, mais importante, uma única função para cada. A letra ``c'',
    por exemplo, para continuar a discussão acima, tem agora sempre o som de ``k'',
    nunca de ``s''; na verdade, foi até renomeada para ``Cá'' [ka], a fim de deixar
    isso mais claro. Pelo mesmo motivo, o ``Cê-cedilha'', ``ç'', é substituído por
    ``s''. E, com isso, acaba-se a ambiguidade entre as consoantes oclusiva velar
    surda [k] e a fricativa alveolar surda [s].

    Analogamente, a letra ``g'' representa apenas a consoante oclusiva velar sonora
        [g] e, como o ``c'', foi renomeada para ``Gá'' [ga], uma vez que o nome ``Gê''
    [\textyogh e], durante séculos, era pronunciado com a fricativa pós-alveolar
    sonora [\textyogh] (i.e. o som da letra ``j'' em português). Assim, por
    exemplo, a palavra ``garagem'', antes escrita com dois ``g'', é, agora
    explicitamente, ``garáje\~y''.

    Seguindo a ordem alfabética, o antigo ``Agá'', ``h'', deixa de ser uma letra
    mal utilizada, essencialmente inútil, e passa a ter o som fricativo glotal
    surdo [h], ou ``Erre" gutural, como é nos demais idiomas da Europa (e.g. nas
    palavras \textit{home}, \textit{heim} e \textit{hjem}, ou seja, ``lar'' em
    inglês, alemão e norueguês, respectivamente). Isso significa que o ``r'' é
    reservado para o tepe alveolar [\textfishhookr] (e.g. em ``para''); e todas as
    palavras que começavam com ``r'', começam com ``h''; e, pela mesma via, aquelas
    que tinham dois ``r'', escreve-se também com ``h''. Por fim, remove-se todos os
    ``h" mudos (e.g. ``hoje''); e, como nas outras letras, renomeia-se o ``Agá''
    para ``Erre'' [\textepsilon hi] e o ``Erre'' para ``Eri'' [\textepsilon
    \textfishhookr i], sinalizando suas novas funções.

    Ao contrário das supracitadas línguas germânicas, entretanto, o ``j'' conserva
    a pronúncia que recebemos dos franceses, não sendo utilizado para o som de
    ``i'' semivogal (como vimos em \textit{hjem}, acima). Esse som, cujo fonema
    denota-se por [j], pertence ao ``y'', que, de maneira análoga ao ``h'', antes
    subutilizado, é agora uma letra muito importante, tendo em vista que o
    brasileiro é um idioma repleto de semivogais.

    Assim, portanto, a indicação das letras semivogais não é nem negligenciada,
    como vinha sendo desde o Acordo Ortográfico de 1990, tampouco se dá pelo
    antiquado ``Trema''. Em contraposição, a nova língua brasileira designa letras
    específicas para esse fim, quais sejam, o ``y'', chamado ``Quasi-i'', e o
    ``w'', ou ``Quasi-u''. Não é necessário mais letras do que essas, porque apenas
    o ``i'' e o ``u'' são semivogais, enquanto o ``a'', o ``e'' e o ``o'' são
    sempre vogais (i.e. elas ``quebram'' a sílaba e não aglutinam-se em ditongos e
    tritongos).

    As semivogais, no entanto, também podem substituir (aparentes) consoantes. É o que verificamos no ``l'' pós-vocálico, que no português brasileiro tem o som do ``u'' semivogal [\textipa{U}], mas não o do ``l'' propriamente dito, isto é, a aproximante lateral alveolar [l] (e.g. no alemão, \textit{hilfe} [\textprimstress h\textsci lf\textschwa], ou no português europeu, ``fiel'' [fi\textprimstress\textepsilon l], com ``l'' pronunciado, cf. no português brasileiro, [fi\textprimstress \texttoptiebar{\textepsilon\textipa{U}}], onde não tem som consonantal). Idem nos dígrafos ``nh'' e ``lh'' (vide o Capítulo \ref{pt.section.digraphs}).

    Outro exemplo é o do ``n'' e o do ``m'' pós-vocálicos, que dão lugar ao ``y'' e ``w'' acentuados com o acento nasal, ``\textasciitilde'', devendo-se à característica distintiva do brasileiro de que vogais seguidas de ``n'' e ``m'' (com uma consoante depois) sempre produzem um som semivocálico residual, que não é perfeitamente capturado por essas duas consoantes, mas sim por aquelas semivogais nasalizadas (viz. ``\~y'', ``\~w'').

    Em praticamente todos os outros idiomas escritos com o alfabeto latino, porém,
    essa ``semivogal residual'' não acontece, então é correto utilizarem o ``n'' e
    o ``m'' pós-vocálicos (e.g. [exemplos]). Mas, como o nosso objetivo é que o
    \textit{brasileiro} seja consistente, devemos substituí-los por semivogais
    nasalizadas.

    Finalmente, as duas últimas letras, ``x'' e ``z'' representam, cada, um único
    som e não mais se confundem entre si nem com o ``s'', ``c'', etc.
    Especificamente, o ``x'' tem, agora, sempre o som da fricativa pós-alveolar
    surda [\textesh] (antigo ``ch''). Já o fonema [z] é grafado pelo ``z'',
    inclusive nas palavras com ``s'' intervocálico (e.g. ``casa''); e não há mais
    ``z'' no final de nenhuma palavra. Desse modo, todas as letras no alfabeto têm
    sua própria função.

    \section{Vogais}
    Como aludido acima, as vogais na língua brasileira são ``a'', ``e'', ``i'',
    ``o'', ``u''; e as semivogais, ``y'' e ``w'' (``Quasi-i'' e ``Quasi-u''). As
    vogais formam hiátos se adjacentes, mas as semivogais aglutinam-se. Por exemplo, nos encontros vocálicos
    \\
    \par ``(em pt-br) palavra com vogal + y'' {[IPA]},
    \par ``(em pt-br) palavra com vogal + w'' {[IPA]},
    \par ``(em pt-br) palavra com y + vogal'' {[IPA]},
    \par ``(em pt-br) palavra com w + vogal'' {[IPA]},
    \par ``(em pt-br) palavra com y + vogal + w'' {[IPA]},
    \par ``(em pt-br) palavra com y + vogal + y'' {[IPA]},
    \par ``(em pt-br) palavra com w + vogal + w'' {[IPA]},
    \par ``(em pt-br) palavra com w + vogal + y'' {[IPA]},
    \\
    \\
    as semivogais e as vogais são uma única sílaba; já nos hiatos
    \\
    \par ``(em pt-br) exemplo de hiato'' [IPA],
    \par ``(em pt-br) exemplo de hiato'' [IPA],
    \par ``(em pt-br) exemplo de hiato'' [IPA],
    \par ``(em pt-br) exemplo de hiato'' [IPA],
    \\
    \\
    as vogais, mesmo adjacentes, não se aglutinam.

    Ademais, porque visamos a consistência fonética (i.e. que se escreva como se
    diz), precisamos distinguir não só entre vogais e semivogais, mas ainda entre
    as agudas, graves e nasais. Denotá-las explicitamente exigiria ou uma letra
    para cada som (como é no Alfabeto Fonético Internacional) ou algum sistema de
    acentuação. A primeira opção não seria nem um pouco prática; a segunda, no
    entanto, também pode tornar-se trabalhosa se não implementada direito.

    Em particular, para evitar excessivos acentos, devemos convencionar uma
    ``pronúncia padrão'' para cada vogal (viz. a mais frequente), e indicar com
    acentos apenas quando a pronúncia for diferente.

    A tabela abaixo define a pronúncia padrão das vogais e semivogais brasileiras:

    \PtTableVowels

    Como pode-se perceber, os fonemas vocálicos são os mesmos do português
    brasileiro tradicional. Então, nesse sentido, excetuando a adição das semivogais, não há
    nada de novo. As pronúncias alternativas, porém, não são as mesmas, conquanto
    sejam mais acuradas do que no português. Expliquemo-las na seguinte seção.

    \subsection{Acentos}
    Para entender as pronúncias alternativas das vogais e semivogais brasileiras,
    convém definirmos, primeiro, os acentos que as indicam:

    \PtTableDiacritics

    As funções dos acentos na Tabela \ref{pt.table.diacritics} são variadas. Mas,
    de maneira geral, servem para: 1) explicitar quando a pronúncia não é a padrão;
    2) indicar a sílaba tônica quando não for autoevidente; 3) diferenciar palavras
    homófonas\footnote{Sendo muito simples a função de diferenciar palavras de mesmo som, opta-se por não explicá-la em detalhes, para não interromper o fluxo deste texto. Porém, a título de exemplo, compare-se o verbo ``há'' (agora escrito ``á'') e o artigo ou a preposição ``a''. Fica claro, aqui, que o acento é o único meio de diferenciar essas homófonas.}.

    \subsubsection{Altura e acentuação}
    A primeira dessas funções é realizada por todos os acentos, exceto a crase.
    Assim, então, quando o acento é agudo, a pronúncia é aguda, mesmo que a
    pronúncia padrão da vogal em questão seja grave; e inversamente se o acento for
    grave.

    O acento nasal também serve para indicar uma pronúncia alternativa. Entretanto,
    nisso difere bastante do que era antes. Na língua brasileira, o acento
    ``\textasciitilde'', não mais chamado ``Til'', faz com que a vogal seja
    pronunciada como seria se fosse seguida de ``n'' ou ``m'', porém de uma maneira
    inteiramente vocálica, ``torcendo'' o som com o nariz, sem a obstrução física
    que caracteriza as consoantes. Isto é, não trata-se de uma vogal ``tendendo''
    ao ``n'' ou ``m'', como é no espanhol ou no italiano, por exemplo, mas daquele
    som nasal \textit{não consonantal}, que é marca do português brasileiro (e.g. [exemplo] [ipa]).

    Acrescenta-se, ainda, que uma vogal nasalizada pode ser tanto aguda quanto
    grave (cf. \textit{adelante} em espanhol e ``adiante'' em português). Em
    teoria, isso requeriria acentos mais específicos, mas, convenientemente, a
    pronúncia aguda ou grave nas vogais nasalizadas é consistente na língua
    brasileira: o ``a'' nasal é sempre grave, enquanto o ``i'', o ``y'', o ``u'' e o ``w'' nasais.

    Já o ``e'' e o ``o'' nunca são
    nasalizados diretamente, porque o som que produziriam, de acordo com a nova
    definição do acento nasal, não ocorre no português brasileiro. Dito isso, se eram seguidos de ``n'' ou ``m'', passam a acompanhar ``\~y'' e ``\~w''
    (de novo, por causa da semivogal residual implícita nesses dígrafos).

    Por fim, sendo evidente que as vogais ``a'', ``e'' e ``o'', seguidas de semivogal nasalizada são sempre graves (e.g. os dígrafos ``ão'', ``em'' e ``om'' em português), convenciona-se, para diminuir a quantidade de acentos, que nelas o ``\textasciigrave'' não se faz necessário.

    \subsubsection{Tonicidade e acentuação}
    Tendo entendido isso, passamos para a segunda função dos acentos, qual seja, a indicação a sílaba tônica. Aqui, de novo, não difere-se muito do português tradicional.

    Temos na língua brasileira apenas palavras oxítonas, paroxítonas e proparoxítonas, cujas sílabas tônicas são, respectivamente, a última, a penúltima e a antepenúltima.

    Dentre essas as mais frequentes são as paroxítonas. E, assim sendo, convenciona-se, tudo o mais constante, que toda palavra de mais de uma sílaba seja paroxítona e não necessita de acento.

    Já as proparoxítonas são, de longe, as mais raras e, por esse motivo, permanecem sempre acentuadas.

    As oxítonas, por sua vez, são relativamente comuns, mas nem sempre requerem acento; isso porque há alguns padrões confiáveis em nossa língua, que nos permitem identificá-las.

    Desse modo, se não indicado explicitamente, são oxítonas (e não acentuadas) todas as palavras terminadas em: ``hr'', ``e'', ``o'', ``y'', ``w'', ``o\~y'', ``o\~w'' (e, igualmente, os plurais, acrescidos de ``s'' no final).

    O último tópico no âmbito da tonicidade refere-se à ``hierarquia'' dos acentos, que é o método com que se identifica a sílaba tônica em palavras que têm múltiplos diacríticos.

    Mais uma vez, a regra é bem simples: o acento nasal forte é a sílaba tônica sempre que aparecer em uma palavra; depois, o acento mais forte é o agudo; e, depois, o grave. O acento nasal (fraco), embora possa coincidir com a penúltima sílaba em paroxítonas, não é, por si só, tônico. E, analogamente, a crase é um diacrítico átono, pois somente consiste na junção de um ``a'' preposição com um ``a'' artigo\footnote{Motivo pelo qual é grafada com ``\textasciidieresis'', sendo também, alternativamente substituída por dois ``a'', se o dispositivo em que se estiver escrevendo não disponibilizar o ``acento duplo''.}, e não modifica nem a pronúncia nem a tonicidade do ``a'' que acentua.

    \subsection{Resumo dos fonemas vocálicos}
    Agora que explicamos a pronúncia de todas as vogais e semivogais brasileiras, podemos resumi-las citando alguns exemplos:
    \newline
    \par [não precisa de 1 exemplo por fonema vocálico]
    \par ``palavra com a [IPA],
    \par ``palavra com á [IPA],
    \par ``palavra com à [IPA],
    \par ``palavra com ã [IPA],
    \par ``palavra com â [IPA],
    \par ``palavra com ä [IPA],
    \par ``palavra com e [IPA],
    \par ``palavra com é [IPA],
    \par ``palavra com è [IPA],
    \par ``palavra com i [IPA],
    \par ``palavra com í [IPA],
    \par ``palavra com ĩ [IPA],
    \par ``palavra com î [IPA],
    \par ``palavra com o [IPA],
    \par ``palavra com ó [IPA],
    \par ``palavra com ò [IPA],
    \par ``palavra com u [IPA],
    \par ``palavra com ú [IPA],
    \par ``palavra com ũ [IPA],
    \par ``palavra com û [IPA],
    \par ``palavra com y [IPA],
    \par ``palavra com \~y [IPA],
    \par ``palavra com w [IPA],
    \par ``palavra com \~w [IPA].
    \newline

    E esses são todos os sons vocálicos no português brasileiro fonético.

    % \subsection{Lei da gravidade vocálica}
    % e $\rightarrow$ i $\rightarrow$ y

    % o $\rightarrow$ u $\rightarrow$ w

    \section{Dígrafos}\label{pt.section.digraphs}
    Enfim, após a apresentação das vogais, só resta tratar brevemente dos dígrafos consonantais\footnote{
        Novamente, os antigos dígrafos vocálicos (viz. vogal seguida de ``n'' ou ``m''), foram substituídos por vogais nasalizadas, ou vogais seguidas de ``\~y'' ou ``\~w'', quando resultam em semivogal residual, como explicado no capítulo anterior.
    }:

    \PtTableDigraphs

    A tabela acima compara os dígrafos no português com sua nova grafia na língua brasileira. Cita-se um exemplo de cada, junto à transcrição para o Alfabeto Fonético Internacional. Comentemo-los abaixo.

    Primeiramente, os dígrafos referentes à consoante fricativa alveolar surda [s] (em português, ``ss'', ``sc'', ``sç'', ``xs'', ``xc'') deixam de sê-lo e são substituídos diretamente pela letra ``s''.

    Seguimos com os dígrafos da ``consoante'' aproximante labiovelar [w] (viz. ``gu'', ``qu''), que, antes ambíguos (cf. na tabela,   ``aguenta'' e ``guerra''), sendo ora pronunciados, ora mudos, agora não são mais utilizados. No português brasileiro fonético, não há vogais mudas; e todas as semivogais são explícitas. Portanto, esses dígrafos não têm mais sentido: se o ``u'' era mudo, é removido; e, se era semivogal, passa a ser ``Quasi-u''\footnote{
        Isso significa que, estritamente falando, o ``Quasi-u'' representa dois fonemas: a vogal quase posterior quase fechada arredondada [\textipa{U}] e
        a semiconsoante (ou semivogal) aproximante labiovelar [w] (podendo, ainda, ser nasalizado).
    }.

    Quanto aos dígrafos da aproximante palatal [j], destaca-se, como já reconhecido pelos fonólogos (e.g. [citar]), que a língua brasileira não dispõe de um som aproximante lateral palatal [\textturny] propriamente dito. E idem para a consoante nasal palatal [\textltailn], isto é, o ``nh'' no português europeu, o ``gn'' em francês e italiano e o ``ñ'' no espanhol (cf.
    montanha, \textit{montagne}, \textit{montagna}, \textit{montaña}). A diferença é sutil, mas notável.

    No Brasil, a nossa pronúncia é muito mais vocálica: aqui, o que temos, no lugar dessas consoantes, são a aproximante lateral alveolo-palatal [\textsubbar{l}\textsuperscript{j}] e a aproximante nasal palatal [\~\j], respectivamente. Em outras palavras, o que em Portugal pronuncia-se com consoante, nós pronunciamos com semivogal.

    Assim, por exemplo, o que causa o ``estalo'' na palavra ``senha'' [\textprimstress se\~\j a] (cf. [\textprimstress se\textltailn\textturna] no português europeu) é que o ``a'' se diz logo após a semivogal nasalizada [\~\j]: ou seja, o ``estalo'' é residual. E, da mesma maneira, na palavra ``batalha'' [ba\textprimstress ta{\textsubbar{l}\textsuperscript{j}}a] (cf. [b\textturna\textprimstress ta\textturny\textturna] no português europeu), há claramente um ``Quasi-i'' residual.

    A grafia adotada para esse último fonema no português brasileiro fonético é ``lly'', sendo semelhante às línguas espanhola e francesa, que escrevem a aproximante lateral palatal com o dígrafo ``ll'' (e.g. [exemplo], [exemplo]).

    As razões para essa escolha são duas: 1) como já dito na seção do alfabeto, a letra ``h'' passou a ter o som do ``Erre'' gutural (i.e. a fricativa glotal surda) e, portanto, agora substituindo o ``r'' no início de palavras e os dois ``r'' intervocálicos, não pode mais ser (mal) utilizada como um ``dígrafo coringa''; 2) e, mais importantemente, o ``lly'' é, de fato, uma perfeita representação do fonema [\textsubbar{l}\textsuperscript{j}], como explicaremos.

    Para entender por que escolheu-se o ``lly'' para grafar o [\textsubbar{l}\textsuperscript{j}], convém contrastarmos os sotaques nordestinos com os demais do Brasil.

    Tomemos, então, a frase ``Minha filha, hoje é dia de ligar para tua tia.'', que contém quase todas as divergências desse dialeto; e comparemos sua notação fonética com a da maior parte dos sotaques brasileiros:
    \\
    \par[
        \textprimstress mĩa
        \textprimstress fi\textsubbar{l}\textsuperscript{j}a,
        \textprimstress\textopeno \textyogh i
        \textepsilon \
        \textprimstress dia
        di
        li\textprimstress gah
        \textprimstress pa\textfishhookr
        a
        \textprimstress tua
        \textprimstress tia.
    ];
    \par[
        \textprimstress mĩa
        \textprimstress fi\textsubbar{l}\textsuperscript{j}a,
        \textprimstress o\textyogh i
        \textepsilon \
        \textprimstress \texttoptiebar{d\textyogh}ia
        \texttoptiebar{d\textyogh}i
        \textsubbar{l}\textsuperscript{j}i\textprimstress gah
        \textprimstress pa\textfishhookr
        a
        \textprimstress tua
        \textprimstress \texttoptiebar{t\textesh}ia.
    ].
    \\

    Aqui, fica claro que no sotaque nordestino, a pronúncia do ``l'' é consistente e o fonema [\textsubbar{l}\textsuperscript{j}] só é utilizado para o ``lh''. Entretanto, é evidente também que, no resto do Brasil, o ``lh'' tem, na verdade, o mesmo som do ``li'' (cf. as palavras ``ilha'' [\textprimstress i\textsubbar{l}\textsuperscript{j}a] e ``família'' [fa\textprimstress mi\textsubbar{l}\textsuperscript{j}a], e compara-se ainda com o italiano, \textit{famiglia} [fa\textprimstress mi\textturny a], onde até escreve-se explicitamente a aproximante lateral palatal com o dígrafo ``gl'').

    Os nordestinos, por assim dizer, recitam o silabário corretamente, enquanto os demais trocam o ``li'' por ``lhi'' (i.e. falando ``la'' [la], ``le'' [le], ``lhi'' [\textsubbar{l}\textsuperscript{j}i], ``lo'' [lo], ``lu'' [lu]). E o mesmo vale para o ``di'' e o ``ti'' (pronunciados ``dji'' [\texttoptiebar{d\textyogh}i] e ``txi'' [\texttoptiebar{t\textesh}i] pela grande maioria dos brasileiros). Com isso, é certo que o ``l'', o ``d'' e o ``t'', seguidos de ``i'' ou ``y'' devem ser novos dígrafos, exceto no brasileiro nordestino fonético.

    Mas, afinal, em que isso nos ajuda com o ``lh''? Ora, eis que o dialeto do nordeste tem, de fato, a chave para escrevermos o [\textsubbar{l}\textsuperscript{j}] sem ``lh''. A fim de pronunciar esse fonema, basta fazer como se fosse falar um ``li'' ``de baiano'' e, rapidamente, interrompê-lo no meio com mais um ``li'': o som resultante é precisamente o ``lhi'', incluindo a semivogal [j] residual; porquanto, não é mera convenção que dois ``l'' e um ``i'' formam um ``lh''.

    [ainda na discussão sobre sotaques, ``hr'']

    [explicar novos dígrafos]


    \section{Exemplos}
     [pseudo-conclusão]

    Concluímos esse manual com alguns exemplos.

    \leftpage
    \section{Ĩtrodusà\~w}

    \newpage
    \section{Awfabétu}
    Comesemus pelu mays bázicu, u awfabétu:
    \BrTableAbc

    Co\~wfóhrmi as tabélas, u novu awfabétu brazileyru (ä escehrda) te\~y vĩtxi i
    cwatru letras, ĩcwãtu u ãtxigu awfabétu pohrtugeys-brazileyru (ä djireyta)
    te\~y vĩtxi i seys. As letras hemovidas fora\~w u ``k'' i u ``q'', pohrce sa\~w
    hedũdãtxis. Dji fatu, a primeyra délas ja éra até na ãtxigwidadji clásica
    critxicada pelus gramátxicus homànus, ci axava\~w-na desnesesarya. A letra
    ``q'', pohr sua veys, foy uma ĩve\~ysà\~w desis mesmus gramátxicus para
    djifere\~ysiahr u so\~w du ``u'' vogaw i du ``u'' semivogaw (cf. as palavras
    \textit{qui} i \textit{cui}). Nós, nu e\~ytãtu, na\~w temus pohr ce fazehr ésa
    djistĩsà\~w, poys a nósa llĩgwa te\~y mays semivogays du ci u Latxî i, alè\~y djisu, mellyóris métodus para espllisitá-las (vehr adjiãtxi). Asî se\~ydu,
    hemovemus du awfabétu acéla letra, desprezada pelus homànus, i, ironicame\~ytxi tãbè\~y ésa, ci ĩve\~ytara\~w.

    Na\~w á nóvas letras nu awfabétu, porè\~y mu\~ytas das ci pehrmanesera\~w
    pasa\~w a tehr nóvas fũso\~ys; i, mays ĩpohrtãtxi, uma única fũsà\~w para cada.
    A letra ``c'', pohr eze\~yplu, para co\~wtxinuahr a djiscusà\~w asima, te\~y
    agóra se\~ypri u so\~w dji ``k'', nũca dji ``s''; na vehrdadji, foy até
    henomiada para ``Ca'' [ka], a fĩ dji deyxahr isu mays claru. Pelu mesmu
    motxivu, u ``Se-sidjillya'', ``ç'', é subistxituidu pohr ``s''. I, co\~w isu,
    acaba-si a ãbigwidadji e\~ytri as co\~wsoãtxis ocluziva velahr suhrda [k] i a
    fricatxiva awveolahr suhrda [s].

    Analogame\~ytxi, a letra ``g'' hepreze\~yta apenas a co\~wsoãtxi ocluziva
    velahr sonóra [g] i, comu u ``c'', foy henomiada para ``Ga'' [ga], uma veys ci
    u nomi ``Je'' [\textyogh e], durãtxi séculus, éra pronũsiadu co\~w a fricatxiva
    pós-awveolahr sonóra [\textyogh] (i.e. o so\~w da letra ``j'' e\~y Pohrtugeys).
    Asî, pohr eze\~yplu, a palavra ``garagem'', ãtxis iscrita co\~w doys ``g'', é,
    agóra espllisitame\~ytxi, ``garaje\~y''.

    Segĩdu a óhrde\~y awfabétxica, u ãtxigu ``Agá'', ``h'', deyxa dji sehr uma
    letra maw utxillizada, ese\~ysiawme\~ytxi inútxiw, i pasa a tehr u so\~w
    fricatxivu glotaw suhrdu [h], ow ``Éhi'' guturaw, comu é nus demays idjiomas da
    Ewrópa (e.g. nas palavras \textit{home}, \textit{heim} i \textit{hjem}, ow
    seja, ``lahr'' e\~y Ĩgleys, Alemà\~w i Noruegeys, hespectxivame\~ytxi). Isu
    significa ci u ``r'' é hezehrvadu para u tépi awveolahr [\textfishhookr] (e.g.
    e\~y ``para''); i todas as palavras ci comesava\~w co\~w ``r'', comésa\~w co\~w
    ``h''; i, pela mesma via, acélas ci txĩa\~w doys ``r'', iscrévi-si tãbè\~y
    co\~w ``h''. Pohr fĩ, hemóvi-si todus us ``h'' mudus (e.g. ``oji''); i, comu
    nas owtras letras, henomeya-si u ``Agá'' para ``Éhi'' [\textepsilon hi] i u
    ``Éhi'' para ``Éri'' [\textepsilon \textfishhookr i], sinalizãdu suas nóvas
    fũso\~ys.

    Aw co\~wtraryu das suprasitadas llĩgwas jehrmànicas, e\~ytretãtu, u ``j''
    co\~wséhrva a pronũsya ci hesebemus dus frãsezis, na\~w se\~ydu utxillizadu
    para u so\~w dji ``i'' semivogaw (comu vimus e\~y \textit{hjem}, asima). Esi
    so\~w, cuju fonema denóta-si pohr [j] pehrte\~ysi aw ``y'', ci, dji maneyra
    análoga aw ``h'', ãtxis sub'utxillizadu, é agóra uma letra mu\~ytu ĩpohrtãtxi,
    te\~ydu e\~y vista ci u Brazileyru é ũ idjioma replétu dji semivogays.

        [falahr da hemosà\~w du ``l'' pós-vocállicu, sitahr ũ eze\~yplu e\~y Alemà\~w, o\~wdji u ``l'' pós-vocállicu é pronũsiadu]

    Asî, pohrtãtu, a ĩdjicasà\~w das letras semivogays na\~w é ne\~y
    negllije\~ysiada, comu vĩa se\~ydu deysdji u Acohrdu Ohrtográficu dji 1990,
    ta\~wpowcu si dá pelu ãtxicwadu ``Trema''. E\~y co\~wtrapozisà\~w, a nóva
    llĩgwa brazileyra dezigna letras espesíficas para esi fĩ, cways seja\~w, u
    ``y'', xamadu ``Cwazi-i'', i u ``w'', ow ``Cwazi-u''. Na\~w é nesesaryu mays
    letras du ci ésas, pohrce apenas u ``i'' i u ``u'' sa\~w semivogays, ĩcwãtu u
    ``a'', u ``e'' i u ``o'' sa\~w se\~ypri vogays (i.e. élas ``cébra\~w'' a sílaba
    i na\~w aglutxina\~w-si e\~y djito\~wgus i trito\~wgus).

    Alè\~y djisu, u ``y'' i u ``w'' sa\~w frecwe\~ytxime\~ytxi ase\~ytuadus co\~w u
    ase\~ytu nazaw, ``\textasciitilde'', i subistxitue\~y u ``n'' i u ``m''
    pós-vocállicus; isu pohrce uma característxica djistxĩtxiva du Brazileyru é ci
    vogays segidas dji ``n'' i ``m'' (co\~w uma co\~wsoãtxi depoys) se\~ypri
    produze\~y ũ so\~w semivocállicu heziduaw, ci na\~w é pehrfeytame\~ytxi
    capturadu pohr ésas duas co\~wsoãtxis, mas sĩ pohr acélas semivogays
    nazallizadas (viz. ``\~y'', ``\~w'').

    E\~y pratxicame\~ytxi todus us owtrus idjiomas iscritus co\~w u awfabétu
    latxinu, porè\~y, ésa ``semivogaw heziduaw'' na\~w aco\~wtési, e\~ytà\~w é
    cohétu utxillizare\~y u ``n'' i u ``m'' pós-vocállicus (e.g. [eze\~yplus]).
    Mas, comu u nósu objetxivu é ci u \textit{Brazileyru} seja co\~wsiste\~ytxi,
    devemus subistxituí-lus pohr semivogays nazallizadas.

    Finawme\~ytxi, as duas úwtximas letras, ``x'' i ``z'' hepreze\~yta\~w, cada, ũ
    únicu so\~w i na\~w mays si co\~wfũde\~y e\~ytri si ne\~y co\~w u ``s'', ``c'',
    etc. Espesificame\~ytxi, u ``x'' te\~y, agóra, se\~ypri u so\~w da fricatxiva
    pós-awveolahr suhrda [\textesh] (ãtxigu ``ch''). Ja u fonema [z] é grafadu pelu
    ``z'', ĩcluzivi nas palavras co\~w ``s'' ĩtehrvocállicu (e.g. ``caza''); i
    na\~w á mays ``z'' nu finaw dji ne\~yuma palavra. Desi módu, todas as letras nu
    awfabétu te\~y sua próprya fũsà\~w.

    \section{Vogays}
    Comu aludjidu asima, as vogays na llĩgwa brazileyra sa\~w ``a'', ``e'', ``i'',
    ``o'', ``u''; i as semivogays, ``y'' i ``w'' (``Cwazi-i'' i ``Cwazi-u''). As
    vogays fóhrma\~w iatus si adjijase\~ytxis, mas as semivogays aglutxina\~w-si. Pohr eze\~yplu, nus e\~yco\~wtrus vocállicus
    \newline
    \par ``(e\~y br-br) palavras co\~w vogaw + y'' {[IPA]},
    \par ``(e\~y br-br) palavras co\~w vogaw + w'' {[IPA]},
    \par ``(e\~y br-br) palavras co\~w y + vogaw'' {[IPA]},
    \par ``(e\~y br-br) palavras co\~w w + vogaw'' {[IPA]},
    \par ``(e\~y br-br) palavras co\~w y + vogaw + w'' {[IPA]},
    \par ``(e\~y br-br) palavras co\~w y + vogaw + y'' {[IPA]},
    \par ``(e\~y br-br) palavras co\~w w + vogaw + w'' {[IPA]},
    \par ``(e\~y br-br) palavras co\~w w + vogaw + y'' {[IPA]},
    \newline

    as semivogays i as vogays sa\~w uma única sílaba; ja nus iatus
    \newline
    \par ``(e\~y br-br) eze\~yplu dji iatu'' [IPA],
    \par ``(e\~y br-br) eze\~yplu dji iatu'' [IPA],
    \par ``(e\~y br-br) eze\~yplu dji iatu'' [IPA],
    \par ``(e\~y br-br) eze\~yplu dji iatu'' [IPA],
    \newline

    as vogays, mesmu adjijase\~ytxis, na\~w si aglutxina\~w.

    Ademays, pohrce vizàmus a co\~wsiste\~ysya fonétxica (i.e. ci si iscreva comu
    si djis), presizàmus djistĩgwihr na\~w só e\~ytri vogays i semivogays, mas aĩda
    e\~ytri as agudas, gravis i nazays. Denotá-las espllisitame\~ytxi ezijiria ow
    uma letra para cada so\~w (comu é nu Awfabétu Fonétxicu Ĩtehrnasyonaw) ow awgû
    sistema dji ase\~ytuasà\~w. A primeyra opsà\~w na\~w é ne\~y ũ powcu prátxica;
    a segũda, nu e\~ytãtu, tãbè\~y pódji tohrnahr-si traballyósa si na\~w
    ĩpleme\~ytada djireytu.

    E\~y pahrtxiculahr, para evitahr esesivus ase\~ytus, devemus co\~wve\~ysyonahr
    uma ``pronũsya padrà\~w'' para cada vogaw (viz. a mays freqwe\~ytxi), i
    ĩdjicahr co\~w ase\~ytus apenas cwãdu a pronũsya fohr djifere\~ytxi.

    A tabéla abayxu defini a pronũsya padrà\~w das vogays i semivogays brazileyras:

    \BrTableVowels

    Comu pódji-si pehrsebehr, us fonemas vocállicus sa\~w us mesmus du Pohrtugeys
    tradjisyonaw. E\~ytà\~w, nesi se\~ytxidu, esetuãdu a adjisà\~w das semivogays,
    na\~w á nada dji novu. As pronũsyas awtehrnatxivas, porè\~y, na\~w sa\~w as
    mesmas, co\~wcwãtu seja\~w mays acuradas du ci nu Pohrtugeys. Espllicemu-las na
    segĩtxi sesà\~w.

    \subsection{Ase\~ytus}
    Para e\~yte\~ydehr as pronũsyas awtehrnatxivas das vogays i semivogays
    brazileyras, co\~wvè\~y definihrmus, primeyru, us ase\~ytus ci as ĩdjica\~w:

    \BrTableDiacritics

    \subsection{E\~yco\~wtrus vocállicus}
    \subsection{Ley da gravidadji vocállica}
    \subsection{Hégras dji ase\~ytuasà\~w}
    % ierahrcia dus aseỹtus
    % crazi

    \section{Djígrafus}

    \BrTableDigraphs

    \section{Eze\~yplus}

\end{bilingualpages}

\newpage
\section{Hezumu}

\newpage
\section{Hefere\~ysyas}

\end{document}