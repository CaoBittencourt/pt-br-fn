\newcommand{\PtTableDiacritics}{
    \begin{longtblr}[
        caption = {Acentos da língua brasileira},
        ]{
        colspec = {X[0, c, m]X[1, c, m]X[0.5, c, m]},
        width = \linewidth,
        rowhead = 1,
        rowfoot = 0
        }
        Acento             & Nome                 & Exemplo \\
        \toprule
        \textasciiacute    & Acento agudo         & dsds    \\
        \textasciigrave    & Acento grave         & dsds    \\
        \textasciitilde    & Acento nasal         & dsds    \\
        \textasciicircum   & Acento nasal forte   & dsds    \\
        \textasciidieresis & Acento duplo (crase) & dsds    \\
        \bottomrule
    \end{longtblr}
}

\newcommand{\BrTableDiacritics}{
    \begin{longtblr}[
        caption = {Ase\~ytus da llĩgwa brazileyra},
        ]{
        colspec = {X[0, c, m]X[1, c, m]X[0.5, c, m]},
        width = \linewidth,
        rowhead = 1,
        rowfoot = 0
        }
        Ase\~ytu           & Nomi                   & Eze\~yplu \\
        \toprule
        \textasciiacute    & Ase\~ytu agudu         & dsds      \\
        \textasciigrave    & Ase\~ytu gravi         & dsds      \\
        \textasciitilde    & Ase\~ytu nazaw         & dsds      \\
        \textasciicircum   & Ase\~ytu nazaw fóhrtxi & dsds      \\
        \textasciidieresis & Ase\~ytu duplu (crazi) & dsds      \\
        \bottomrule
    \end{longtblr}
}